\documentclass[
	% -- opções da classe memoir --
	12pt,					% tamanho da fonte
	openright,				% capítulos começam em pág ímpar (insere página vazia caso preciso)
	oneside,					% para impressão em verso e anverso. Oposto a oneside
	a4paper,					% tamanho do papel. 
	% -- opções da classe abntex2 --
	%chapter=TITLE,			% títulos de capítulos convertidos em letras maiúsculas
	%section=TITLE,			% títulos de seções convertidos em letras maiúsculas
	%subsection=TITLE,		% títulos de subseções convertidos em letras maiúsculas
	%subsubsection=TITLE,	% títulos de subsubseções convertidos em letras maiúsculas
	% -- opções do pacote babel --
	english,					% idioma adicional para hifenização
	%french,					% idioma adicional para hifenização
	%spanish,				% idioma adicional para hifenização
	brazil					% o último idioma é o principal do documento
	]{abntex2}

% ---------------------
% Pacotes OBRIGATÓRIOS
% ---------------------
\usepackage{lmodern}			% Usa a fonte Latin Modern			
\usepackage[T1]{fontenc}		% Selecao de codigos de fonte.
\usepackage[utf8]{inputenc}		% Codificacao do documento (conversão automática dos acentos)
\usepackage{lastpage}			% Usado pela Ficha catalográfica
\usepackage{indentfirst}		% Indenta o primeiro parágrafo de cada seção.
\usepackage{color}				% Controle das cores
\usepackage{graphicx,graphicx}	% Inclusão de gráficos
\usepackage{epsfig,subfig}		% Inclusão de figuras
\usepackage{microtype} 			% Melhorias de justificação
\usepackage{verbatim}           % Comentarios de mais de uma linha
% ---------------------
		
% ---------------------
% Pacotes ADICIONAIS
% ---------------------
\usepackage{listings}                   % https://www.overleaf.com/learn/latex/Code_listing
\usepackage{lipsum}						% Geração de dummy text
\usepackage{amsmath,amssymb,mathrsfs}	% Comandos matemáticos avançados 
\usepackage{setspace}  					% Para permitir espaçamento simples, 1 1/2 e duplo
\usepackage{tabularx} 					% Para poder ter tabelas com colunas de largura auto-ajustável
\usepackage{multirow}                   % 
\usepackage{afterpage} 					% Para executar um comando depois do fim da página corrente
\usepackage{url} 						% Para formatar URLs (endereços da Web)
\usepackage[table]{xcolor}
\usepackage{float}                      % Para colocar imagens exatamente na posição desejada
\usepackage{mathtools}                  % Símbolos matemáticos
\usepackage{colortbl}                   % Colorir células de tabelas
\usepackage{color}
\usepackage{array}
% ---------------------


% ---------------------
% Pacotes de CITAÇÕES
% ---------------------
\usepackage[brazilian,hyperpageref]{backref}	% Paginas com as citações na bibl
\usepackage[alf]{abntex2cite}					% Citações padrão ABNT (alfa)
%\usepackage[num]{abntex2cite}					% Citações padrão ABNT (numericas)
% ---------------------

% Configurações de CITAÇÕES para abntex2
\include{extras/conf_citacoes}

% Inclusão de dados para CAPA e FOLHA DE ROSTO (título, autor, orientador, etc.)
% ---
% Informações de dados para CAPA e FOLHA DE ROSTO
% ---
\titulo{Análise de Sentimentos Temporal a partir de Avaliações Online de Hotéis no Google Maps}
\autor{Emerson Almeida Matos}
\local{Santo André - SP}
\data{Novembro de 2024} % TODO
\orientador{Alexandre Donizeti Alves}
%\coorientador{Fulano Nome do Coorientador}
\instituicao{%
  Universidade Federal do ABC -- UFABC
  \par
  Centro de Matemática, Computação \& Cognição 
  \par
  Bacharelado em Ciência da Computação}
\tipotrabalho{Projeto de Graduação em Computação}
% O preambulo deve conter o tipo do trabalho, o objetivo,
% o nome da instituição e a área de concentração
\preambulo{\textbf{Projeto de Graduação em Computação} apresentado como parte dos requisitos necessários para a obtenção do Título de Bacharel em Ciência da Computação.}
% ---


% Inclui Configurações de aparência do PDF Final
\include{extras/conf_pdf}

% Inclui Configurações para o listing
% ---
% CONFIGURAÇÕES DE PACOTES
% ---

\usepackage{listingsutf8}
%New colors defined below
\definecolor{codegreen}{rgb}{0,0.6,0}
\definecolor{codegray}{rgb}{0.5,0.5,0.5}
\definecolor{codepurple}{rgb}{0.58,0,0.82}
\definecolor{backcolour}{rgb}{0.95,0.95,0.92}
%Code listing style named "mystyle"
\lstdefinestyle{mystyle}{
	backgroundcolor=\color{backcolour},
	commentstyle=\color{codegreen},
	keywordstyle=\color{magenta},
	numberstyle=\tiny\color{codegray},
	stringstyle=\color{codepurple},
	basicstyle=\ttfamily\footnotesize,
	breakatwhitespace=false,
	breaklines=true,
	captionpos=b,
	keepspaces=true,
	numbers=left,
	numbersep=5pt,
	showspaces=false,
	showstringspaces=false,
	showtabs=false,
	tabsize=2
}

\lstset{
style=mystyle,
literate=
	{á}{{\'a}}1
{à}{{\`a}}1
{ã}{{\~a}}1
{é}{{\'e}}1
{ê}{{\^e}}1
{í}{{\'i}}1
{ó}{{\'o}}1
{õ}{{\~o}}1
{ú}{{\'u}}1
{ü}{{\"u}}1
{ç}{{\c{c}}}1,
extendedchars=true,}


% O tamanho da identação do parágrafo é dado por:
\setlength{\parindent}{1.3cm}

% Controle do espaçamento entre um parágrafo e outro:
\setlength{\parskip}{0.2cm}  % tente também \onelineskip

% ---------------------
% Compila o indice
% ---------------------
\makeindex
% ---------------------

%%%%%%%%%%%%%%%%%%%%%%%%%%%
%%  INICIO DO DOCUMENTO  %%
%%%%%%%%%%%%%%%%%%%%%%%%%%%
\begin{document}

% Retira espaço extra obsoleto entre as frases.
\frenchspacing

% ----------------------------------------------------------
% ELEMENTOS PRÉ-TEXTUAIS (Capa, Resumo, Abstract, etc.)
% ----------------------------------------------------------
\pretextual

% Capa
\include{pretextual/capa}

% Folha de rosto (o * indica que haverá a ficha bibliográfica)
\imprimirfolhaderosto*

\begin{comment}

% Imprimir Ficha Catalografica
\include{pretextual/catalografica}

% Inserir Folha de Aprovação
\include{pretextual/assinaturas}

% Dedicatória
% ---
% Dedicatória
% ---
\begin{dedicatoria}
   \vspace*{\fill}
   \centering
   \noindent
   \textit{Dedico este trabalho de conclusão de curso à minha família, amigos e professores, cujo apoio, incentivo e orientação foram fundamentais para a realização deste projeto. A todos vocês, meu sincero agradecimento.} \vspace*{\fill}
\end{dedicatoria}
% ---


% Agradecimentos
% % ---
% Agradecimentos
% ---
\begin{agradecimentos}

Primeiramente, agradeço aos meus pais, pela paciência, amor incondicional e apoio constante em cada etapa da minha vida. Sem vocês, essa conquista não seria possível.

Agradeço ao meu orientador, Prof. Alexandre Donizeti Alves, por sua orientação, paciência e dedicação ao longo deste projeto. Suas valiosas sugestões e ensinamentos foram fundamentais para a realização deste trabalho.

Gostaria de agradecer também aos colegas e amigos que, garantidamente, contribuíram para o desenvolvimento deste projeto, seja mediante conversas construtivas, suporte emocional ou companheirismo ao longo da graduação. Em especial, agradeço aos amigos da UFABC que compartilharam essa jornada acadêmica comigo e fizeram dessa experiência algo mais leve e agradável.

Agradeço à Universidade Federal do ABC por proporcionar um ambiente acadêmico de alta qualidade e por oferecer as ferramentas necessárias para o desenvolvimento acadêmico e pessoal.

Por fim, agradeço a todos que, direta ou indiretamente, me apoiaram nessa caminhada. Este trabalho é fruto de uma longa jornada, e sou profundamente grato a cada um que contribuiu para que ele se tornasse realidade.

Muito obrigado!

\bigskip

\begin{flushright}
Emerson Almeida Matos
\end{flushright}
\end{agradecimentos}
%% ---

% Epígrafe
% ---
% Epígrafe
% ---
\begin{epigrafe}
	\vspace*{\fill}
	\begin{flushright}
		\textit{``Todo o conhecimento começa com o espanto.''}\\
		– Aristóteles
	\end{flushright}
\end{epigrafe}
% ---


% Resumo e Abstract
% ---
% RESUMOS
% ---

% RESUMO em português
\setlength{\absparsep}{18pt} % ajusta o espaçamento dos parágrafos do resumo
\begin{resumo}

Neste projeto, foi desenvolvido um processo de análise de sentimentos temporal aplicado às avaliações online de hotéis no \textit{Google Maps}. Com o crescente volume de dados disponíveis em plataformas de avaliação, como \textit{Google Maps}, torna-se essencial entender como as opiniões de usuários evoluem ao longo do tempo, auxiliando novos clientes a tomarem decisões mais informadas. Para isso, propõe-se a utilização de modelos de Processamento de Linguagem Natural, incluindo modelos baseados em \textit{BERT} e outros modelos de linguagem de última geração, como \textit{GPT-3.5}, \textit{OpenChat} e \textit{Vicuna}, para classificar os sentimentos expressos nas avaliações. Além disso, são analisadas as variações temporais dessas avaliações, permitindo identificar tendências e mudanças no comportamento dos usuários ao longo do tempo. Os resultados mostram a predominância de avaliações positivas e revelam padrões temporais nas opiniões dos clientes. A análise de sentimentos temporal oferece uma visão valiosa para empresas e usuários, proporcionando percepções sobre a qualidade do serviço ao longo do tempo.

 \textbf{Palavras-chaves}: Análise de Sentimentos, Grandes Modelos de Linguagem, Google Maps.
\end{resumo}

% ABSTRACT in english
\begin{resumo}[Abstract]
 \begin{otherlanguage*}{english}
In this project, a temporal sentiment analysis process was developed, applied to online hotel reviews on Google Maps. With the increasing volume of data available on review platforms like Google Maps, it becomes essential to understand how user opinions evolve over time, assisting new customers in making more informed decisions. To achieve this, the use of Natural Language Processing models is proposed, including models based on BERT and other state-of-the-art language models, such as GPT-3.5, OpenChat, and Vicuna, to classify the sentiments expressed in the reviews. Additionally, the temporal variations of these reviews are analyzed, allowing for the identification of trends and changes in user behavior over time. The results show the predominance of positive reviews and reveal temporal patterns in customer opinions. Temporal sentiment analysis provides valuable insights for businesses and users, offering perceptions of service quality over time.
   \vspace{\onelineskip}
 
   \noindent 
   \textbf{Keywords}: Sentiment analysis, Large Language Models, Google Maps.
 \end{otherlanguage*}
\end{resumo}

% Lista de ilustrações
\pdfbookmark[0]{\listfigurename}{lof}
\listoffigures*
\cleardoublepage

% Lista de tabelas
\pdfbookmark[0]{\listtablename}{lot}
\listoftables
\cleardoublepage

% Lista de abreviaturas e siglas
\begin{siglas}
	\item[API] Application Programming Interface (Interface de Programação de Aplicação)
	\item[BERT] Bidirectional Encoder Representations from Transformers
	\item[CETIC] Centro Regional de Estudos para o Desenvolvimento da Sociedade da Informação
	\item[COVID-19] Coronavirus disease 2019
	\item[GPT] Generative Pre-trained Transformer
	\item[LLaMa] Large Language Model Meta AI
	\item[LLM] Large Language Models (Grandes Modelos de Linguagem)
	\item[LMSYS] Large Model Systems Organization
	\item[LSTM] Long short-term memory
	\item[NLP] Natural language processing
	\item[PLN] Processamento de linguagem natural
	\item[SVM] Support Vector Machine
	\item[VADER] Valence Aware Dictionary and sEntiment Reasoner
\end{siglas}

% Lista de símbolos
%\begin{simbolos}
%	\item[$ \Gamma $] Letra grega Gama
%	\item[$ \Lambda $] Lambda
%	\item[$ \zeta $] Letra grega minúscula zeta
%	\item[$ \in $] Pertence
%\end{simbolos}
\end{comment}

% Inserir o SUMÁRIO
\pdfbookmark[0]{\contentsname}{toc}
\tableofcontents*
\cleardoublepage

% ----------------------------------------------------------
% ELEMENTOS TEXTUAIS (Capítulos)
% ----------------------------------------------------------
\textual
% Elementos textuais com numeração arábica
\pagenumbering{arabic}
% Reinicia a contagem do número de páginas
\setcounter{page}{1}


% Inclui cada capitulo do Trabalho
% ----------------------------------------------------------
% Introdução 
% Capítulo sem numeração, mas presente no Sumário
% ----------------------------------------------------------

\chapter[Introdução]{Introdução}
\label{cap:intro}
% \addcontentsline{toc}{chapter}{Introdução}

\begin{comment}
Neste capítulo precisamos:
\begin{itemize}
    \item Introduzir o contexto
    \item Definir o que entendemos como avaliação de hotéis
    \item Apresentar em linhas gerais quais métodos são usados para detectar fake news, qual o estado da arte atual e quais são suas limitações.
    \item Descrever a nossa proposta e objetivos
    \item Descrever a estrutura do relatório.
\end{itemize}

\end{comment}

Com os avanços da tecnologia, o acesso à informação tem sido cada vez mais facilitado entre a população geral. Antes, as fontes de informações mais frequentes eram o conhecimento de prévio de indivíduos mais experientes ou conteúdos impressos \cite{jayathilake2021critical} . Uma pesquisa feita pela \citeonline{cetic2019pesquisa} indica que 3 entre cada 4 brasileiros possuem acesso à internet e esses entrevistados também responderam que realizaram algum tipo de consulta online, onde 28\% procurou por informações referentes a viagens e acomodações, um índice baixo se comparado aos 57\% que respondeu ter buscado por informações sobre produtos e serviços, esse acesso pelo meio digital tende a ser aumentar quando olhamos para a população da Geração Z, sendo a geração que tende a possuir uma maior facilidade com a tecnologia \cite{sfodera2022technology}.

Quando surge o interesse em realizar viagens, sites como Booking.com, Tripadvisor e AirBnb, são alguns dos sites referencia em relação à reserva de hospedagem em hotel \citeonline{best2022HostSites} relacionadas a hospedagem. É feita uma busca de informações de pessoas que já frequentaram os locais, onde cada usuário da plataforma preenche as avaliações conforme o modelo disponível em cada plataforma, partindo de uma avaliação objetiva dividida por categorias, onde o usuário é questionado por área de interesse de modo a atribuir um valor número como nota para aquela experiência, em muitos casos, existindo um bloco de texto disponível para relatar a experiência em sua estadia local. As plataformas citadas costumam ter, além das avaliações dos usuários, informações próprias onde a plataforma preenche, incluindo informações de preços por diárias e distâncias de atrações.

Essas informações costumam direcionar o internauta e o auxiliam em sua tomada de decisão, avaliações online são a segunda fonte mais confiável de informações, logo atrás das recomendações de familiares e amigos \cite{chatterjee2020drivers}, porém qual a validade dessas informações? como essas avaliações variam durante o tempo? esse hotel está tendo avaliações melhores agora do que no mesmo período no ano passado? esse local está sendo avaliado negativamente por conta de algum evento atípico? mesmo registrando uma avaliação ruim, o usuário que a submeteu possui outra avaliação registrada no passado?

Contudo, as plataformas nem sempre expõem as informações claras e objetivas, dificultando a tomada de decisões para o internauta, que possui o objetivo de escolher um local que contemple suas necessidades durante seu período de viagem, afinal, o internauta não tem interesse em ser surpreendido que as avaliações dispostas estavam equivocadas e as informações mais recentes tenham sido ocultadas devido às tendências negativas por conta da organização por nível de popularidade nas plataformas.

Atualmente o Google possui uma ferramenta de busca muito popular, presente em diversos aparelhos celulares, e a frase "restaurantes proximos a mim" é a frase mais utilizada para busca de locais nos ultimos cinco anos \cite{li20213}, ela também dispõe de uma funcionalidade de avaliação de locais \cite[Google Maps]{googleMaps2022Blog}, conforme descrito no próprio blog deles:

\begin{quote}
    Para quem quer conhecer lugares novos, as avaliações deixadas por usuários do Google são uma rica fonte de conhecimento e informações locais. Elas ajudam a encontrar estabelecimentos e empresas que se encaixam no gosto de cada um — seja uma padaria com o melhor bolo de laranja sem glúten da cidade ou um restaurante de bairro com deliciosa música ao vivo.
\end{quote}

As avaliações do Google Maps podem ser escritas por qualquer usuário do Google Maps, o que pode ser um ponto negativo se considerarmos que algumas das plataformas supracitadas exigem que o internauta necessariamente tenha se hospedado em no hotel que deseja avaliar.

\begin{comment}
O presente relatório está estruturado da seguinte forma: o capítulo~\ref{cap:justificativa} apresenta… o capítulo~\ref{cap:fund_teorica} ... O capítulo~\ref{cap:metodologia} ..., o capítulo~\ref{cap:resultados} ... O capítulo~\ref{cap:conclusao}

Demonstração de citação: o software de análise foi desenvolvido na linguagem Python~\cite{van1995python}, usando as bibliotecas Pandas~\cite{mckinney2010data} e Scikit-learn~\cite{scikit-learn}.
\end{comment}

\section{Justificativa}

Estamos vivendo em um mundo com muitas informações disponíveis e de fácil acesso, porém esta cada vez mais difícil interpretar essas informações e entender de modo a enriquecer o nosso conhecimento atual sobre determinado assunto.

No cenário em que procuramos informações sobre possíveis estadias é importante saber selecionar as avaliações com as informações mais próximas da realidade atual do estabelecimento, queremos evitar o viés intrínseco contido nas avaliações para conseguir escolher a estadia que possui o maior valor para o internauta.

\section{Estrutura da Monografia}

O presente trabalho está dividido da seguinte forma:
\begin{enumerate}
    \item O primeiro capítulo é a Introdução, que contextualiza o tema e sobre a proposta do trabalho.
    \item O segundo capítulo apresenta os objetivos.
    \item O terceiro capítulo apresenta informações sobre a metodologia.
\end{enumerate}

%\include{capitulos/justificativa}
%\chapter{Cronograma}
\label{cap:cronograma}

% \begin{figure}[h]
    % \center
    % \includegraphics[width=17cm]{figs/cronograma.png}
% \end{figure}

Inicialmente, será criado um corpus para análise com conteúdo suficiente para obter variação significativa nos sentimentos com o avanço do tempo, para ser possível identificar visualmente essas variações por hotel, tópico e por região.

Após essa etapa será necessário realizar o tratamento dos dados e o treinamento do modelo BERT para realizar a classificação do corpus.

Com o treinamento concluído, espera-se realizar o emparelhamento de adjetivo-substantivo e a avaliação de sentimento deles e de todo o corpus.

Com os resultados em mãos, será realizada um agrupamento por região e hotéis, visando mostrar de forma gráfica como essas avaliações têm variado temporalmente.

De forma resumida, o seguinte cronograma de atividades deverá ser considerado:

\begin{table}[H]
\centering
{\footnotesize
\doublespacing
\begin{tabular}{|p{6.1cm}|l|l|l|l|l|l|l|l|l|}
\hline
\multirow{}{}{ATIVIDADES}
& \multicolumn{4}{c|}{PGC II} & \multicolumn{5}{c|}{PGC III} \\ \cline{2-10}
& Set  & Out  & Nov  & Dez & Jan & Fev & Mar & Abr & Mai \\ \hline
1. Conjunto de Dados                   & X  & X &  &  &   &   &   &  &       \\ \hline
2. Pré-processamento dos Dados                         &   & X  & X  &   &  &  &  &    &         \\ \hline
3. Análise de Sentimentos               
 &  &   & X & X & X & X &  &  &  \\ 
 \hline
4. Análise Temporal dos Dados                                   &     &     &     &     & X   & X   & X    &  X  &       \\ \hline
5. Escrita de Relatórios                                    &     &     &  X  &  X  &    &    & X   & X   &      \\ \hline
6. Defesa                                    &     &     &    &    &    &    &    &    & X     \\ \hline

\end{tabular}}
\caption{Cronograma de Atividades}
\label{tab:cronograma}
\end{table}

\chapter{Fundamentação Teórica}
\label{cap:fund_teorica}

A análise de sentimentos é uma área da ciência da computação e do processamento de linguagem natural (PLN) que tem como objetivo identificar e extrair informações sobre sentimentos, opiniões e atitudes expressas em textos escritos.
Neste capítulo abordaremos alguns dos conceitos necessários para o desenvolvimento e entendimento do projeto, do campo de estudo de processamento de linguagem natural, mais especificamente a análise de sentimentos. Também iremos discorrer sobre a funcionalidade de avaliação de estabelecimentos exposta pelo Google Maps.

\section{Google Maps}
\label{cap:fund_teorica:sec:google_maps}

O Google Maps é uma das ferramentas de mapeamento e navegação mais populares do mundo, desenvolvida e mantida pela Google. Lançado em 2005, o Google Maps rapidamente se tornou uma ferramenta essencial para milhões de pessoas ao redor do mundo, oferecendo mapas detalhados, direções de navegação, informações sobre o tráfego em tempo real e uma ampla gama de recursos adicionais, como visualização de imagens de satélite, avaliações de estabelecimentos comerciais e muito mais.

Um dos recursos interessantes do Google Maps é o programa "Local Guides". Este programa foi introduzido pela Google para incentivar os usuários a contribuir com informações adicionais sobre lugares e estabelecimentos em suas comunidades locais. Os "Local Guides" são usuários voluntários que compartilham avaliações, fotos, informações sobre horários de funcionamento, preços e outros detalhes úteis sobre restaurantes, lojas, atrações turísticas e outros pontos de interesse.

Os "Local Guides" são incentivados pela Google a contribuir com conteúdo de qualidade, e em troca, eles podem ganhar pontos, níveis e até mesmo recompensas, como armazenamento adicional no Google Drive, acesso antecipado a novos recursos do Google Maps e até mesmo convites para eventos exclusivos. Essa gamificação do processo de contribuição ajuda a manter os usuários engajados e motivados a compartilhar informações úteis.

Além disso, os "Local Guides" desempenham um papel importante na melhoria contínua da precisão e da utilidade do Google Maps. Ao fornecer informações atualizadas e detalhadas sobre os locais, eles ajudam outros usuários a tomar decisões informadas sobre onde ir e o que fazer. Essa contribuição colaborativa cria uma comunidade global de usuários que trabalham juntos para tornar o Google Maps uma ferramenta ainda mais poderosa e abrangente.

Em resumo, o Google Maps é uma ferramenta essencial para navegação e descoberta de lugares, enquanto os "Local Guides" são membros voluntários da comunidade que contribuem com informações valiosas para melhorar a precisão e utilidade do serviço. Juntos, eles desempenham um papel crucial na criação de uma experiência de mapeamento e navegação mais rica e envolvente para todos os usuários.

\subsection{Avaliação de estabelecimentos}
\label{cap:fund_teorica:sec:google_maps:subsec:avaliacao_estabelecimentos}


A funcionalidade de avaliação de estabelecimentos no Google Maps é uma ferramenta poderosa que oferece aos usuários uma maneira de compartilhar suas experiências e opiniões sobre diferentes locais, desde restaurantes e cafés até hotéis e pontos turísticos. Essa funcionalidade permite que os usuários forneçam feedback valioso sobre a qualidade do serviço, a atmosfera do local, a qualidade dos produtos e uma infinidade de outros aspectos.

Uma das principais vantagens dessa funcionalidade é a capacidade de ajudar outros usuários a tomar decisões informadas ao escolherem um estabelecimento para visitar. As avaliações e comentários fornecidos por outros usuários podem servir como uma referência confiável, permitindo que as pessoas tenham uma noção melhor do que esperar de um determinado local antes de decidirem visitá-lo.

Além disso, o sistema de avaliação do Google Maps geralmente é muito acessível e fácil de usar. Os usuários podem rapidamente atribuir uma classificação de estrelas e deixar um comentário detalhado sobre sua experiência com apenas alguns cliques, ajudando a manter uma base de dados robusta e atualizada sobre os estabelecimentos locais.

No entanto, é importante reconhecer que nem todas as avaliações podem ser completamente imparciais ou precisas. Algumas podem ser tendenciosas ou baseadas em experiências individuais que podem não refletir a experiência média dos clientes. Portanto, é sempre aconselhável considerar várias avaliações e fontes de informações ao tomar uma decisão.

Em resumo, a funcionalidade de avaliação de estabelecimentos no Google Maps oferece aos usuários uma maneira conveniente de compartilhar e acessar informações sobre locais locais, ajudando a tornar as experiências de viagem e exploração mais informadas e satisfatórias.
% falar sobre o google maps e as avaliações e que avaliações de idioma diferente são traduzidas. traduzido por google
% preprocessamento
% pesquisar sobre os algoritmos de analise de sentimento https://www.lume.ufrgs.br/bitstream/handle/10183/140910/000991520.pdf?sequence=1
% https://www.teses.usp.br/teses/disponiveis/59/59143/tde-31102018-174346/publico/mateuscorrigida.pdf
% pesquisar sobre classificação de avaliações

\section{Nuvem de Palavras}
\label{cap:fund_teorica:sec:word_cloud}
% TODO
%
Uma tecnica bastante popular é o agrupamento dos tokens para que seja posteriormente realizada uma representação visual dos dados, onde o tamanho do token representa sua frenquência em um dado corpus, e sua coloração pode ser util para representar o sentimento do token. Essa representação gráfica é nomeanda \emph{wordcloud}, ou nuvem de palavras.

O modelo de gráfico de nuvens de palavras é uma técnica de visualização frequentemente utilizada na análise de sentimentos para representar as palavras mais frequentes em um conjunto de dados de texto, com o tamanho das palavras proporcional à sua frequência de ocorrência. Neste contexto, as palavras são tipicamente coloridas de acordo com seu sentimento associado, o que facilita a identificação visual das palavras mais relevantes e das tendências sentimentais no texto.

Essa abordagem é valiosa na análise de sentimentos, pois permite uma rápida identificação de palavras-chave associadas a diferentes emoções, como positivas, negativas ou neutras. Ao visualizar um gráfico de nuvens de palavras, os analistas podem extrair insights sobre as percepções dos usuários e as principais tendências de sentimentos presentes nos comentários ou avaliações de um determinado produto, serviço ou tópico.

O modelo de gráfico de nuvens de palavras é uma ferramenta poderosa para a visualização e interpretação de dados de texto em análises de sentimentos, proporcionando uma representação intuitiva e acessível das informações linguísticas subjacentes.

\section{Análise de Sentimento}
\label{cap:fund_teorica:sec:analise_sentimento}
% 10.1145/1014052.1014073 Mining and Summarizing Customer Reviews
A análise de sentimentos é uma das áreas de pesquisa em processamento de linguagem natural, que visa utilizar recursos computacionais para estudar os sentimentos, opiniões e emoções de uma determinada sentença em relação a um alvo \cite{Bellini2019}.

Uma opinião é composta de dois componentes chaves, formando o par \emph{(g,s)}: um alvo \emph{g} e um sentimento \emph{s} relacionado a esse alvo. E como destacado em \cite{Machado2018}, \emph{g} pode se referir a qualquer aspecto da entidade ou a ela própria e sentimento \emph{s} pode ser positivo, negativo ou neutro, ou ser representado por um valor numérico que representa a sua intensidade, denominado como polarização do sentimento ou opinião.



\section{Modelos}
\label{cap:fund_teorica:sec:modelos}


Neste projeto foram utilizados \emph{Large Language Models}(LLM) pré-treinados, como o GPT (Generative Pre-trained Transformer), OpenChat e Vicuna, além de modelos baseados na arquitetura \emph{BERT} (Bidirectional Encoder Representations from Transformers). Esses modelos representam o estado-da-arte em PLN e têm demonstrado um desempenho significativo em uma variedade de tarefas, incluindo análise de sentimentos.

\subsection[Bert]{Bert}
\label{cap:fund_teorica:sec:modelos:subsec:bert}

% TODO bert
% % https://arxiv.org/pdf/1909.10649.pdf
Um modelo baseado em \emph{BERT}(\textit{Bidirectional Encoder Representations from Transformers}) é um tipo de modelo de linguagem pré-treinado desenvolvido pelo Google. \emph{BERT} é um modelo de aprendizado profundo que utiliza a arquitetura de transformers para capturar representações bidirecionais de palavras em um texto. Isso significa que ele é capaz de entender o contexto de uma palavra com base em suas palavras vizinhas, tanto anteriores quanto posteriores, em uma sentença.

Os modelos BERT são uma família de modelos de linguagem baseados em transformers que foram pré-treinados em grandes corpora de texto. Eles são capazes de capturar o contexto bidirecional das palavras em uma frase, o que os torna especialmente eficazes para tarefas de análise de sentimentos, onde o contexto é crucial para determinar o sentimento expresso.

O treinamento de \emph{BERT} é realizado em grandes quantidades de dados de texto, onde o modelo aprende a prever palavras em uma frase ou sentença com base no contexto global da sentença. Como resultado, \emph{BERT} é capaz de capturar nuances e complexidades da linguagem natural de uma maneira mais eficaz do que modelos mais simples.

Depois de pré-treinado em um grande corpus de texto, um modelo \emph{BERT} pode ser ajustado para tarefas específicas, como análise de sentimentos, classificação de texto, resposta a perguntas, entre outras, por meio de um processo chamado de ajuste fino (\textit{fine-tuning}). Durante o ajuste fino, o modelo é treinado em um conjunto de dados rotulado específico para a tarefa em questão, adaptando suas representações aprendidas durante o pré-treinamento para a tarefa específica.

Os modelos baseados em \emph{BERT} tornaram-se amplamente populares e são frequentemente usados como base para uma variedade de tarefas de processamento de linguagem natural devido à sua capacidade de capturar contextos complexos e produzir resultados de alta qualidade.

\subsection[Large Language Models]{LLM}
\label{cap:fund_teorica:sec:modelos:subsec:LLM}
% TODO LLM
% https://arxiv.org/pdf/2307.06435.pdf

O GPT é um modelo baseado em transformadores que aprende representações de linguagem generalizadas a partir de grandes quantidades de texto não rotulado. Ele é capaz de gerar texto coerente e relevante, e pode ser adaptado para tarefas específicas, como análise de sentimentos, através do treinamento supervisionado em conjuntos de dados rotulados.

O OpenChat \cite{wang2023openchat}  é um modelo de conversação baseado em transformadores que foi pré-treinado em uma grande quantidade de dados de conversação. Ele é capaz de gerar respostas contextuais e relevantes em conversas humanas, o que pode ser útil para entender o contexto em avaliações de hotéis e extrair sentimentos.

O Vicuna \cite{vicuna2023} é outro modelo baseado em transformadores que foi projetado especificamente para análise de sentimentos. Ele foi treinado em uma grande variedade de conjuntos de dados de sentimentos e é capaz de realizar tarefas de classificação de sentimentos com alta precisão.

% topicos? https://sol.sbc.org.br/livros/index.php/sbc/catalog/download/4/177/354-1?inline=1
% hugging face?
% todo

\chapter{Trabalhos Relacionados}
\label{cap:trabalhos_relacionados}

% todo

\section{Análise em larga escala da evolução temporal de tópicos obtidos do twitter basado em Apache Spark}
\label{cap:trabalhos_relacionados:sec:braulio}

No trabalho de \citeonline{vinces2022}, foi realizada uma análise abrangente da evolução temporal dos tópicos extraídos do \textit{Twitter}, empregando o \textit{Apache Spark} para processamento distribuído. A modelagem de tópicos em redes sociais, como o \textit{Twitter}, desempenha um papel crucial em diversas aplicações, como detecção de notícias e análise de sentimentos. O objetivo principal foi obter resultados comparáveis ou superiores ao modelo \textit{Latent Dirichlet Allocation} (LDA), com foco na escalabilidade e eficiência computacional, utilizando modelos \textit{BERT} para aprimorar a extração de características textuais.

Durante a pesquisa, foram explorados conceitos teóricos relacionados à modelagem de tópicos, incluindo \textit{Word Embeddings} e métricas de avaliação, com ênfase na análise temporal dos tópicos identificados. A metodologia adotada envolveu um pipeline de trabalho estruturado em cinco etapas, visando uma abordagem eficaz para a análise dos dados do \textit{Twitter} ao longo do tempo. Os resultados experimentais, impulsionados pela utilização de modelos \textit{BERT} pré-treinados, foram fundamentais para as conclusões alcançadas, destacando a relevância e o impacto do estudo no campo da Inteligência Artificial e textit{Big Data}.


\section{Análise do atendimento bancário relacionado a espólio na plataforma consumidor.gov.br}
\label{cap:trabalhos_relacionados:sec:desterro}

O trabalho de \citeonline{desterro2023} teve como objetivo analisar reclamações e avaliações relacionadas a espólios registradas contra instituições bancárias brasileiras na plataforma \textit{Consumidor.gov.br}, focando especialmente no período da pandemia de COVID-19. Para isso, foram utilizadas técnicas de Processamento de Linguagem Natural (PLN) para analisar os sentimentos das avaliações dos consumidores e identificar os principais tópicos das reclamações. Os dados foram extraídos por meio de \textit{web scraping}, empregando o modelo \textit{XLM-T} para a análise de sentimentos e o algoritmo de clusterização \textit{K-Means} para identificar os tópicos predominantes.

A pesquisa revelou que os representantes de clientes falecidos enfrentam principalmente problemas relacionados ao encerramento de contas, documentação para inventário e atendimento nas agências bancárias. Além disso, foi observada uma predominância de sentimentos negativos nas avaliações dos consumidores. Esses resultados são importantes pois podem ajudar as instituições bancárias a identificar e resolver problemas recorrentes no atendimento aos representantes de clientes falecidos, melhorando assim a qualidade do serviço prestado.

O estudo de Desterro destaca a utilidade de aplicar técnicas de PLN na análise de opinião de consumidores, demonstrando que essas técnicas podem ser valiosas para empresas que buscam aprimorar seu atendimento ao cliente. A análise detalhada e sistemática das reclamações pode fornecer  percepções importantes para a implementação de melhorias nos processos e na comunicação com os clientes, especialmente em situações sensíveis como o gerenciamento de espólios.

\chapter{Metodologia}
\label{cap:metodologia}

\section{Conjunto de Dados}

Será utilizada como fonte de dados as avaliações disponíveis online no idioma português brasileiro na plataforma do Google Maps.

A lista de hotéis será obtida através da \emph{Google Places API} buscando por região. A consulta a API terá a função de fornecer uma lista de locais com suas respectivas informações simplificadas, como entrada recebe o texto que será utilizado para realizar a busca.

O texto utilizado para a busca dos hotéis é definida no seguinte modelo “frase + região”. Por exemplo, na frase “hotéis próximos ao Cristo Redentor”, a API deverá retornar informações básicas de hotéis cadastrados na plataforma próximos à região do Cristo Redentor.

A lista de hotéis será obtida pelas regiões das principais atrações do Brasil segundo~\cite{googleFlights2022destinos}. Essa lista é composta com base nas análises conduzidas pelo Google, que considera a quantidade de menções online e as suas avaliações recebidas na plataforma.

Posteriormente essa lista irá servir como entrada para um \emph{Web Scraping} para buscar as avaliações onlines disponíveis no Google Maps, realizadas pelos \emph{Local Guides}~\cite{google2022localguides} registrados na plataforma. Serão consideradas apenas as avaliações que possuírem texto, e os dados obtidos serão exportados para um arquivo do tipo CSV.

\section{Pré-processamento dos Dados}

Nessa etapa utilizaremos o BERT (\emph{Bidirectional Encoder Representations from Transformers}), sendo este um modelo de representação de linguagem desenvolvido por pesquisadores do Google e que teve seu código aberto em 2018~\cite{hugoZanini2021mediu}. O texto será tokenizado utilizando a biblioteca e um modelo pré treinado denominado "BERTimbau Base"~\cite{souza2020bertimbau}.

Será realizado também o emparelhamento de Adjetivo-Substantivo visando identificar o sentimento por tópicos contidos na avaliação analisada. Para isso será utilizada a biblioteca spaCy~\cite{montani2022spacy}. 

\section{Análise de Sentimentos}

Com o modelo treinado e os dados prontos para serem classificados, o modelo irá atribuir uma classificação da seguinte forma:
\begin{itemize}
    \item Uma classificação para o nosso documento, que aqui consideramos cada avaliação individualmente.
    \item Uma classificação para sentença, que será cada frase da  avaliação.
    \item Uma classificação para cada tópico identificado.
\end{itemize}

\section{Análise Temporal dos Dados}

Após a análise de sentimentos, utilizaremos os resultados agrupados por períodos de modo a conseguir visualizar em modo gráfico as avaliações divididas conforme explicitado na seção anterior para um grupo de hotéis.  

\section{Cronograma}

Inicialmente, será criado um corpus para análise com conteúdo suficiente para obter variação significativa nos sentimentos com o avanço do tempo, para ser possível identificar visualmente essas variações por hotel, tópico e por região.

Após essa etapa será necessário realizar o tratamento dos dados e o treinamento do modelo BERT para realizar a classificação do corpus.

Com o treinamento concluído, espera-se realizar o emparelhamento de adjetivo-substantivo e a avaliação de sentimento deles e de todo o corpus.

Com os resultados em mãos, será realizada um agrupamento por região e hotéis, visando mostrar de forma gráfica como essas avaliações têm variado temporalmente.

De forma resumida, o seguinte cronograma de atividades deverá ser considerado:

\begin{table}[H]
\centering
{\footnotesize
\doublespacing
\begin{tabular}{|p{6.1cm}|l|l|l|l|l|l|l|l|l|}
\hline
\multirow{}{}{ATIVIDADES}
& \multicolumn{4}{c|}{PGC II} & \multicolumn{5}{c|}{PGC III} \\ \cline{2-10}
& Set  & Out  & Nov  & Dez & Jan & Fev & Mar & Abr & Mai \\ \hline
1. Conjunto de Dados                   & X  & X &  &  &   &   &   &  &       \\ \hline
2. Pré-processamento dos Dados                         &   & X  & X  &   &  &  &  &    &         \\ \hline
3. Análise de Sentimentos               
 &  &   & X & X & X & X &  &  &  \\ 
 \hline
4. Análise Temporal dos Dados                                   &     &     &     &     & X   & X   & X    &  X  &       \\ \hline
5. Escrita de Relatórios                                    &     &     &  X  &  X  &    &    & X   & X   &      \\ \hline
6. Defesa                                    &     &     &    &    &    &    &    &    & X     \\ \hline

\end{tabular}}
\caption{Cronograma de Atividades}
\label{tab:cronograma}
\end{table}


\chapter{Resultados}
\label{cap:resultados}

% todo

\chapter{Conclusões}
\label{cap:conclusao}

\section{Contribuições}
\label{cap:conclusao:sec:contribuicoes}
% TODO

Este projeto apresentou uma análise abrangente de sentimentos temporais baseada em avaliações online de hotéis no \textit{Google Maps}, utilizando modelos de linguagem de última geração, como \textit{BERT}, \textit{Vicuna}, \textit{GPT 3.5} e \textit{OpenChat}. Através desta investigação, foi possível verificar que, para tarefas de classificação de sentimento a nível de documento, modelos mais simples e menos complexos como o \textit{BERT} são capazes de oferecer resultados satisfatórios, principalmente quando consideramos o custo-benefício em termos de tempo de execução e recursos computacionais.

Também fica evidente que não há necessidade de utilização de modelos proprietários, pois modelos não proprietários tem desempenho satisfatórios na tarefa de classificação de sentimento, porém com a necessidade de possuir uma infraestrutura adequada para que possa ser possível a execução dos mesmos.

Além disso, também fica evidente que a abordagem utilizada no projeto é suficientemente boa para ser utilizada em plataformas de recomendação de hospedagem, e podem inclusive ser aprimorada para fornecer mais detalhes sobre os pontos de destaque tanto positivos como negativos de um dado hotel, evitando a necessidade dos usuários de realizar de forma manual a investigação e revisão das avaliações, que em muitos cenários tomaria um tempo considerável para que seja realizado por humanos.

\section{Limitações}
\label{cap:conclusao:sec:limitacoes}

Uma das limitações desse trabalho é em relação aos recursos disponíveis para execução dos modelos abertos. Por conta do ambiente de execução utilizado possuir apenas 12 GB de VRAM não foi possível realizar testes com modelos maiores, e por isso não foi possível adicionar modelos maiores para comparação e como consequência não temos informação suficiente para podermos afirmar que modelos maiores sejam mais ou menos eficientes na execução da tarefa proposta. Outro grande impacto é em relação ao texto (\textit{prompt}) utilizado como início da conversação com cada modelo, em que alguns modelos podem ser mais sensíveis ao texto utilizado para inciar a tarefa de classificação do texto da avaliação, se faz necessário um estudo mais aprofundado para entender os possíveis impactos da utilização de diferentes textos com diferentes modelos.

Fica difícil notar variação de sentimento temporal com o conjunto de dados utilizado, principalmente por conta da característica de serem hotéis de grande qualidade e por esse motivo o volume de dados com sentimento positivo foi esmagadoramente superior aos outros. Outra forma de aprimorar seria realizar uma análise mais granular com relação a data que não foi possível por conta da limitação da plataforma do \textit{Google Maps}, ou utilizar um conjunto de hotéis de menor qualidade, em que é esperado que as avaliações com sentimento positivo sejam mais equivalentes, em termos de quantidade, com relação as outras.

\section{Trabalhos Futuros}
\label{cap:conclusao:sec:trab_futuros}

Para trabalhos futuros podemos explorar a análise de impacto do texto inicial utilizado para que o modelo realize a tarefa, e entender a sensibilidade dos modelos para com as alterações no conteúdo utilizado como estimulo para início do processamento, bem como o impacto da utilização de modelos iguais porém com tamanho de parâmetros diferentes. Além disso, pode ser explorado também a utilização de um corpus com uma variação maior de nota de classificação para tentar identificar uma variação de sentimento maior com relação ao tempo, ou até uma possível exploração com maior detalhe utilizando o poder de processamento dos grandes modelos de linguagem, fazendo com que a tarefa não seja uma simples atribuição de etiqueta de sentimento, mas também identificação do sentimento das entidades presentes nos documentos, isto é, o sentimento especifico para uma dada característica do hotel. Também poderia inverter e no lugar de olhar o conjunto de avaliações de um dado estabelecimento passar a olhar para o conjunto de avaliações de uma lista de \textit{local guides} e entender a relação de sentimento de suas avaliações e a nota atribuída e também a aplicação de técnicas mais avançadas de pré-processamento poderiam fornecer resultados ainda mais precisos e abrangentes.



% ----------------------------------------------------------
% ELEMENTOS PÓS-TEXTUAIS (Referências, Glossário, Apêndices)
% ----------------------------------------------------------
\postextual
% Referências bibliográficas
\bibliography{bibliografia}


% Glossário (Consulte o manual)
%\glossary

% Apêndices
% % ----------------------------------------------------------
% Apêndices
% ----------------------------------------------------------

% ---
% Inicia os apêndices
% ---
\begin{apendicesenv}

% Imprime uma página indicando o início dos apêndices
\partapendices

% ----------------------------------------------------------
\chapter{Primeiro Apêndice}
% ----------------------------------------------------------

\begin{figure}
  \centering
  \includegraphics[width=0.75\textwidth]{figs/exploratoria/distribuicao_ano_avaliacao.png}
  \caption{Distribuição do número de avaliação por ano}
  \label{img:dist_ano_avaliacao}
\end{figure}

\chapter{Segundo Apêndice}
\label{apendice:lista_completa_hoteis}

link de planilhas com lista de hotéis
% https://docs.google.com/spreadsheets/d/1I1vKzFdVqG5FYIGJYRUwwkeI_wbzCThij92SbBmrdKE/edit?usp=sharing
\end{apendicesenv}


% Anexos
%\include{postextual/anexos}

% Índice remissivo (Consultar manual)
%\phantompart
%\printindex

\end{document}
