% ---
% RESUMOS
% ---

% RESUMO em português
\setlength{\absparsep}{18pt} % ajusta o espaçamento dos parágrafos do resumo
\begin{resumo}

Este projeto pretende desenvolver um processo de análise de sentimentos temporal aplicado às avaliações online de hotéis no \textit{Google Maps}. Com o crescente volume de dados disponíveis em plataformas de avaliação, como \textit{Google Maps}, há a necessidade de entender como as opiniões de usuários evoluem ao longo do tempo, auxiliando novos clientes a tomarem decisões mais informadas. Para tanto, o trabalho propõe a utilização de modelos de Processamento de Linguagem Natural (PLN), incluindo modelos baseados em \textit{BERT} e outros modelos de linguagem de última geração, como \textit{GPT-3.5},\textit{OpenChat} e \textit{Vicuna}, para realizar a classificação dos sentimentos expressos nas avaliações. Além disso, são analisadas as variações temporais das avaliações, possibilitando identificar tendências e mudanças no comportamento dos usuários ao longo do tempo. A metodologia inclui a coleta automatizada de dados do \textit{Google Maps}, pré-processamento das informações e a aplicação dos modelos PLN para a extração dos sentimentos. Os resultados evidenciam a predominância de avaliações positivas e revelam padrões temporais nas opiniões dos clientes. Conclui-se que a análise de sentimentos temporal oferece uma visão valiosa para empresas e usuários, proporcionando percepções sobre a qualidade do serviço ao longo do tempo.



 \textbf{Palavras-chaves}: Análise de sentimento, Processamento de Linguagem Natural, Grandes modelos de linguagem, Google Maps, GPT, Vicuna, OpenChat, BERT.
\end{resumo}

% ABSTRACT in english
\begin{resumo}[Abstract]
 \begin{otherlanguage*}{english}
    This project aims to develop a process for temporal sentiment analysis applied to online reviews of hotels available on Google Maps. With the increasing volume of data on review platforms like Google Maps, there is a need to understand how user opinions evolve over time, helping new customers make more informed decisions. Therefore, this work proposes the use of Natural Language Processing (NLP) models, including \textit{BERT-based} models and other state-of-the-art language models like GPT-3.5, OpenChat and Vicuna, to classify the sentiments expressed in the reviews. Additionally, the temporal variations in reviews are analyzed, allowing the identification of trends and shifts in user behavior over time. The methodology includes automated data collection from Google Maps, data pre-processing, and the application of NLP models to extract sentiments. The results show a predominance of positive reviews and reveal temporal patterns in customer opinions. It is concluded that temporal sentiment analysis provides valuable insights for companies and users, offering an understanding of service quality over time.

   \vspace{\onelineskip}
 
   \noindent 
   \textbf{Keywords}: Sentiment analysis, Natural Language Processing, Large language models, Google Maps, GPT, Vicuna, OpenChat, BERT.
 \end{otherlanguage*}
\end{resumo}