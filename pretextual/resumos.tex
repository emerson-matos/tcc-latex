% ---
% RESUMOS
% ---

% RESUMO em português
\setlength{\absparsep}{18pt} % ajusta o espaçamento dos parágrafos do resumo
\begin{resumo}

Neste projeto, foi desenvolvido um processo de análise de sentimentos temporal aplicado às avaliações online de hotéis no \textit{Google Maps}. Com o crescente volume de dados disponíveis em plataformas de avaliação, como \textit{Google Maps}, torna-se essencial entender como as opiniões de usuários evoluem ao longo do tempo, auxiliando novos clientes a tomarem decisões mais informadas. Para isso, propõe-se a utilização de modelos de Processamento de Linguagem Natural, incluindo modelos baseados em \textit{BERT} e outros modelos de linguagem de última geração, como \textit{GPT-3.5}, \textit{OpenChat} e \textit{Vicuna}, para classificar os sentimentos expressos nas avaliações. Além disso, são analisadas as variações temporais dessas avaliações, permitindo identificar tendências e mudanças no comportamento dos usuários ao longo do tempo. Os resultados mostram a predominância de avaliações positivas e revelam padrões temporais nas opiniões dos clientes. A análise de sentimentos temporal oferece uma visão valiosa para empresas e usuários, proporcionando percepções sobre a qualidade do serviço ao longo do tempo.

 \textbf{Palavras-chaves}: Análise de Sentimentos, Grandes Modelos de Linguagem, Google Maps.
\end{resumo}

% ABSTRACT in english
\begin{resumo}[Abstract]
 \begin{otherlanguage*}{english}
In this project, a temporal sentiment analysis process was developed, applied to online hotel reviews on Google Maps. With the increasing volume of data available on review platforms like Google Maps, it becomes essential to understand how user opinions evolve over time, assisting new customers in making more informed decisions. To achieve this, the use of Natural Language Processing models is proposed, including models based on BERT and other state-of-the-art language models, such as GPT-3.5, OpenChat, and Vicuna, to classify the sentiments expressed in the reviews. Additionally, the temporal variations of these reviews are analyzed, allowing for the identification of trends and changes in user behavior over time. The results show the predominance of positive reviews and reveal temporal patterns in customer opinions. Temporal sentiment analysis provides valuable insights for businesses and users, offering perceptions of service quality over time.
   \vspace{\onelineskip}
 
   \noindent 
   \textbf{Keywords}: Sentiment analysis, Large Language Models, Google Maps.
 \end{otherlanguage*}
\end{resumo}