\chapter{Metodologia}
\label{cap:metodologia}

\section{Conjunto de Dados}

Será utilizado como fonte de dados as avaliações disponíveis online no idioma português brasileiro na plataforma do Google Maps, visando simplificar a etapa de tratamento e uniformização dos dados.

A lista de hotéis será obtida através da \emph{Google Places API} buscando por região. A consulta a API terá a função de fornecer uma lista de locais com suas respectivas informações simplificadas, como entrada recebe o texto que será utilizado para realizar a busca.

O texto utilizado para a busca dos hotéis é definida no seguinte modelo “frase + região”, por exemplo, na frase “hotéis próximos ao Cristo Redentor”, a API deverá retornar informações básicas de hotéis cadastrados na plataforma próximos à região do Cristo Redentor.

A lista de hotéis será obtida pelas regiões das principais atrações do Brasil segundo \cite{googleFlights2022destinos}. Essa lista é composta com base nas análises conduzidas pelo Google, que considera a quantidade de menções online e as suas avaliações recebidas na plataforma.

Posteriormente essa lista irá servir como entrada para um \emph{Web Scraping} para buscar as avaliações onlines disponíveis no Google Maps, realizadas pelos Local Guides\cite{google2022localguides} registrados na plataforma, somente utilizaremos as avaliações que possuírem texto, e os dados obtidos serão exportados para um arquivo do tipo CSV.

\section{Pré-processamento dos Dados}

Nessa etapa utilizaremos o BERT, sendo este um modelo de representação de linguagem desenvolvido por pesquisadores do Google e que teve seu código aberto em 2018 \cite{hugoZanini2021mediu}. O texto será tokenizado utilizando a biblioteca e um modelo pré treinado BERTimbau Base \cite{souza2020bertimbau}.

Será realizado também o emparelhamento de Adjetivo-Substantivo visando identificar o sentimento por tópicos contidos na avaliação analisada, para essa função a biblioteca spaCy \cite{montani2022spacy} será utilizada. 

\section{Análise de Sentimentos}


Com o modelo treinado e os dados prontos para serem classificados, o modelo irá atribuir uma classificação da seguinte forma:
\begin{itemize}
    \item Uma classificação para o nosso documento, que aqui consideramos cada avaliação individualmente.
    \item Uma classificação para sentença, que será cada frase da nossa avaliação.
    \item Uma classificação para cada tópico identificado.
\end{itemize}

\section{Análise Temporal dos Dados}

Realizada a análise de sentimentos utilizaremos os resultados agrupados por períodos de modo a conseguir visualizar em modo gráfico as avaliações dividias conforme explicitado na seção anterior para um grupo de hotéis com uma grande variação e os que possuem uma pequena variação no tempo, bem como uma comparação dos dados das agrupadas por região.  

\section{Cronograma}

Inicialmente será montado o corpus para análise com conteúdo suficiente para obter variação significativa nos sentimentos com o avanço do tempo, para ser possível identificar visualmente essas variações por hotel, tópico e por região.

Após essa etapa será necessário realizar o tratamento dos dados e o treinamento do modelo BERT para realizar a classificação do corpus.

Com o treinamento concluído espera-se realizar o emparelhamento de adjetivo-substantivo e a avaliação de sentimento deles e de todo o corpus.

Com os resultados em mãos será realizada um agrupamento por região e hotéis, visando mostrar de forma gráfica como essas avaliações têm variado temporalmente.