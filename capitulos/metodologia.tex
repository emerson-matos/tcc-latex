\chapter{Metodologia}
\label{cap:metodologia}

Este capítulo apresenta a metodologia adotada neste trabalho. A seção \ref{cap:metodologia:sec:conjunto_dados:sec:escolha_conjunto} descreve o processo de escolha dos dados; a seção \ref{cap:metodologia:sec:conjunto_dados:sec:coleta} descreve como foi realizada a coleta dos dados.
% ; a seção \ref{subsec:pre_processamento} descreve // TODO; a seção \ref{subsec:analise_sentimentos} descreve // TODO; a seção \ref{subsec:analise_temporal} descreve // TODO


\section{Conjunto de Dados}
\label{cap:metodologia:sec:conjunto_dados}

\subsection{Escolha do conjunto}
\label{cap:metodologia:sec:conjunto_dados:sec:escolha_conjunto}

O conjunto de dados foi selecionado considerando os hotéis com informações disponíveis no Google Maps e considerando todo o território brasileiro.

A primeira etapa consiste em identificar os hotéis distribuídos no território brasileiro agrupados por estado, por meio do serviço exposto do Google Maps, a iteração foi realizada utilizando a biblioteca escrita em java, \citeonline{javaGoogleMapsService2022} que está disponível publicamente no repositório no \citeonline{github2023}.

Para identificar os diversos hotéis, o primeiro passo foi escrever um \emph{script} Kotlin \citeonline{scriptKotlinBuscarHoteis} que interage com a biblioteca supracitada para realizar a busca da lista de hotéis de forma automatizada, realizando buscas de forma sequencial considerando todos os estados brasileiros e armazenando os dados obtidos.

O objetivo nesta etapa é identificar e obter informações dos hotéis que estão disponíveis para consulta, e para essa tarefa utilizaremos a interface de \emph{Text Search Query}, informações como o nome do hotel, localização e identificador do hotel na plataforma são disponibilizadas por essa interface. Nessa interface conseguimos realizar a busca passando informações que servirão para polarizar a busca, como idioma de interesse, coordenadas e diâmetro limite para ser considerado e dessa forma obter uma busca com viés dos parâmetros fornecidos \cite{placesSearchText2023}.

\lstinputlisting[language=java,caption=Interface Google Maps para realizar busca na plataforma]{extras/code/googlemaps-place-search.kt}

Após obter a lista com as informações básicas o segundo momento é obter informações mais detalhadas do estabelecimento, para isso utilizaremos a interface de \emph{Place Details} para poder obter essas informações na plataforma, dentre as informações expostas por esta interface possuímos algumas que são de nosso interesse como, por exemplo, número de avaliações disponíveis, classificação atual na plataforma e quantidade de avaliações, entre outras são obtidas nessa etapa.

\lstinputlisting[language=java,caption= Interface Google Maps]{extras/code/googlemaps-place-details.kt}

Foi possível identificar 815 hotéis distribuídos em todo o território nacional, desde hotéis de grandes redes e com disponibilidade de reservas do tipo \emph{All Inclusive} até hotéis pouco conhecidos com menos de 10 avaliações na plataforma.

% Estados com a maior quantidade de hoteis
% \begin{itemize}
%    \item RJ: 71
%    \item SP: 67
%    \item PA: 62
%    \item MG: 61
%    \item RR/PI/MS: 60
% \end{itemize}

% Estados com a maior quantidade de reviews

% \begin{itemize}
% 	\item RJ: 231947 (4,22 com 71 hotéis)
% 	\item SP: 169586 (4,21 com 67 hotéis)
% 	\item BA: 107716 (4,34 com 22 hotéis)
% 	\item MG: 107067 (4,35 com 61 hotéis)
% 	\item PB: 68188 (4,36 com 54 hotéis)
% \end{itemize}


% Média da classificação dos hoéis por Estado

% \begin{itemize}
% 	\item RS: 4,70 com 9493 avaliações no total (2 hotéis)
% 	\item AL: 4,64 com 33345 avaliações no total (8 hotéis)
% 	\item AC: 4,60 com 1696 avaliações no total (1 hotel)
% 	\item AM: 4,53 com 45509 avaliações no total (46 hotéis)
% 	\item SC: 4,50 com 32621 avaliações no total (7 hotéis)
% \end{itemize}

% TODO adicionar analise da lista de hoteis graficos ou meta informações

Por conta da grande variedade de hotéis, o objetivo foi restringir a análise a um grupo de 70 hotéis dentre os disponíveis, distribuídos da seguinte forma:

\begin{itemize}
  \item 10 com nota menor ou igual à 4, independente da sua classificação.
  \item 10 com mais de 2000 avaliações.
  \item 10 com nota igual ou superior a 4,8 e com mais de 2000 avaliações.
  \item 20 com notas próximas a 4 e com mais de 2000 avaliações.
  \item 20 outros com pacote \emph{All Inclusive}.
\end{itemize}


% A lista de hotéis será obtida pelas regiões das principais atrações do Brasil segundo~\cite{googleFlights2022destinos}. Essa lista é composta com base nas análises conduzidas pelo Google, que considera a quantidade de menções online e as suas avaliações recebidas na plataforma.


\subsection{Coleta das avaliações}
\label{cap:metodologia:sec:conjunto_dados:sec:coleta}

Após definida a lista de hotéis seguimos para a de coleta das avaliações.

Já existe um script escrito em Python \cite{gaspa93scrapper2023} que dado uma \emph{URL} do Google Maps, utilizando Selenium, obtém um determinado número de avaliações, o \emph{script} porém estava defasado e pendente de atualização, um dos \emph{forks} existentes \citeonline{ryuuzakescrapper2023} estava também pouco mais atualizado, porém ainda não funcional.

Utilizando esses projetos como base escrevi então um script para realizar o trabalho de obter as avaliações dos hotéis, o script era escrito em Python e as suas principais dependências eram de \cite{selenium2023}, \citeonline{harris2020array}, \citeonline{jeffreback20226702671} e \citeonline{richardson2007beautiful}. Utilizando \citeonline{merkel2014docker} para executar o script era possível rodar o script de forma concorrente, utilizando Selenium executando diversos drivers do Google Chrome, porém o desempenho não era muito agradável, uma execução demorava aproximadamente 3h para executar com metade para metade dos hotéis e o sucesso da execução do script dependia diretamente da imutabilidade da interface do Google Maps, por conta da forma com que o script estava escrito e da estrutura das páginas do Google Maps.

O script tinha uma dependência muito forte com tags HTML e classes CSS da página dos locais no Google Maps ...

% todo
Será utilizada como fonte de dados as avaliações disponíveis online no idioma português brasileiro na plataforma do Google Maps.

A lista de hotéis será obtida através da \emph{Google Places API} buscando por região. A consulta a API terá a função de fornecer uma lista de locais com suas respectivas informações simplificadas, como entrada recebe o texto que será utilizado para realizar a busca.

O texto utilizado para a busca dos hotéis é definida no seguinte modelo “frase + região”. Por exemplo, na frase “hotéis próximos ao Cristo Redentor”, a API deverá retornar informações básicas de hotéis cadastrados na plataforma próximos à região do Cristo Redentor.

A lista de hotéis será obtida pelas regiões das principais atrações do Brasil segundo~\cite{googleFlights2022destinos}. Essa lista é composta com base nas análises conduzidas pelo Google, que considera a quantidade de menções online e as suas avaliações recebidas na plataforma.

Posteriormente essa lista irá servir como entrada para um \emph{Web Scraping} para buscar as avaliações onlines disponíveis no Google Maps, realizadas pelos \emph{Local Guides}~\cite{google2022localguides} registrados na plataforma. Serão consideradas apenas as avaliações que possuírem texto, e os dados obtidos serão exportados para um arquivo do tipo CSV.


\subsection{Pré-processamento}
\label{subsec:pre_processamento}


Nessa etapa utilizaremos o BERT (\emph{Bidirectional Encoder Representations from Transformers}), sendo este um modelo de representação de linguagem desenvolvido por pesquisadores do Google e que teve seu código aberto em 2018~\cite{hugoZanini2021mediu}. O texto será tokenizado utilizando a biblioteca e um modelo pré-treinado denominado "BERTimbau Base"~\cite{souza2020bertimbau}.

Será realizado também o emparelhamento de Adjetivo-Substantivo visando identificar o sentimento por tópicos contidos na avaliação analisada. Para isso será utilizada a biblioteca spaCy~\cite{montani2022spacy}. 

\section{Análise dos dados}

\subsection{Análise de Sentimentos}
\label{subsec:analise_sentimentos}

Com o modelo treinado e os dados prontos para serem classificados, o modelo irá atribuir uma classificação da seguinte forma:
\begin{itemize}
    \item Uma classificação para o nosso documento, que aqui consideramos cada avaliação individualmente.
    \item Uma classificação para sentença, que será cada frase da  avaliação.
    \item Uma classificação para cada tópico identificado.
\end{itemize}

\subsection{Análise Temporal}
\label{subsec:analise_temporal}
Após a análise de sentimentos, utilizaremos os resultados agrupados por períodos de modo a conseguir visualizar em modo gráfico as avaliações divididas conforme explicitado na seção anterior para um grupo de hotéis.
