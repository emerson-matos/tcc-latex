\chapter{Metodologia}
\label{cap:metodologia}

\section{Conjunto de Dados}

\subsection{Coleta de dados}
Será utilizada como fonte de dados as avaliações disponíveis online no idioma português brasileiro na plataforma do Google Maps.

A lista de hotéis será obtida através da \emph{Google Places API} buscando por região. A consulta a API terá a função de fornecer uma lista de locais com suas respectivas informações simplificadas, como entrada recebe o texto que será utilizado para realizar a busca.

O texto utilizado para a busca dos hotéis é definida no seguinte modelo “frase + região”. Por exemplo, na frase “hotéis próximos ao Cristo Redentor”, a API deverá retornar informações básicas de hotéis cadastrados na plataforma próximos à região do Cristo Redentor.

A lista de hotéis será obtida pelas regiões das principais atrações do Brasil segundo~\cite{googleFlights2022destinos}. Essa lista é composta com base nas análises conduzidas pelo Google, que considera a quantidade de menções online e as suas avaliações recebidas na plataforma.

Posteriormente essa lista irá servir como entrada para um \emph{Web Scraping} para buscar as avaliações onlines disponíveis no Google Maps, realizadas pelos \emph{Local Guides}~\cite{google2022localguides} registrados na plataforma. Serão consideradas apenas as avaliações que possuírem texto, e os dados obtidos serão exportados para um arquivo do tipo CSV.

\subsection{Pré-processamento dos Dados}

Nessa etapa utilizaremos o BERT (\emph{Bidirectional Encoder Representations from Transformers}), sendo este um modelo de representação de linguagem desenvolvido por pesquisadores do Google e que teve seu código aberto em 2018~\cite{hugoZanini2021mediu}. O texto será tokenizado utilizando a biblioteca e um modelo pré treinado denominado "BERTimbau Base"~\cite{souza2020bertimbau}.

Será realizado também o emparelhamento de Adjetivo-Substantivo visando identificar o sentimento por tópicos contidos na avaliação analisada. Para isso será utilizada a biblioteca spaCy~\cite{montani2022spacy}. 

\section{Análise dos dados}

\subsection{Análise de Sentimentos}

Com o modelo treinado e os dados prontos para serem classificados, o modelo irá atribuir uma classificação da seguinte forma:
\begin{itemize}
    \item Uma classificação para o nosso documento, que aqui consideramos cada avaliação individualmente.
    \item Uma classificação para sentença, que será cada frase da  avaliação.
    \item Uma classificação para cada tópico identificado.
\end{itemize}

\subsection{Análise Temporal dos Dados}

Após a análise de sentimentos, utilizaremos os resultados agrupados por períodos de modo a conseguir visualizar em modo gráfico as avaliações divididas conforme explicitado na seção anterior para um grupo de hotéis.