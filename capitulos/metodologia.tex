\chapter{Metodologia}
\label{cap:metodologia}

\section{Conjunto de Dados}

Será utilizado como fonte de dados as avaliações disponíveis online no idioma português na plataforma do Google Maps, visando simplificar a etapa de tratamento e uniformização dos dados.

A lista de hoteis será obtida através da \emph{Google Places API} buscando por região. A consulta a API terá a função de fornecer uma lista de locais com suas respectivas informações simplificadas, como entrada recebe o texto que será utilizado para realizar a busca.

O texto utilizado para a busca dos hotéis é definida no seguinte modelo “frase + região”, por exemplo, na frase “hotéis próximos ao Cristo Redentor”, a API deverá retornar informações básicas de hotéis cadastrados na plataforma próximos a região do Cristo Redentor.

A lista de hotéis será obtida pelas regiões das principais atrações do Brasil segundo \cite{googleFlights2022destinos}. Essa lista é composta com base nas analises feitas pelo Google, que considera a quantidade de menções online e as suas avaliações recebidas na plataforma.

Posteriormente essa lista irá servir como entrada para um \emph{Web Scraping} para buscar as avaliações onlines disponíveis no Google Maps, realizadas pelos Local Guides\cite{google2022localguides} registrados na plataforma.

\section{Pré-processamento dos Dados}



\section{Análise de Sentimentos}

% documento -> review
% sentenca -> cada frase
% topicos -> sentimentos por topicos

\section{Análise Temporal dos Dados}



\section{Cronograma}
