\chapter{Metodologia}
\label{cap:metodologia}

\section{Conjunto de dados}

Utilizaremos unicamente como fonte de dados as avaliações disponíveis na plataforma do Google Maps, visando simplificar a etapa de tratamento e uniformização dos dados.

Utilizaremos a \emph{Google Places API} para automatizar o processo de identificação de hotéis por região. A API exposta tem por objetivo fornecer uma lista de locais com suas respectivas informações de forma simplificada e recebe como entrada um texto que será utilizado na busca.

A princípio realizaremos buscas de hotéis por região no modelo “frase + região”, por exemplo, na frase “hotéis próximos à Santo André”, a API irá retornar informações básicas de hotéis cadastrados na plataforma próximos de Santo André.

Posteriormente essa lista irá servir como entrada para um \emph{Web Scraping} para buscar as informações de avaliações disponíveis no Google Maps, realizadas pelos Local Guides\cite{google2022localguides} registrados na plataforma.

\section{Pré-processamento dos Dados}



\section{Análise de Sentimentos}



\section{Análise Temporal dos Dados}



\section{Cronograma}
