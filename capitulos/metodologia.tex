\chapter{Metodologia}
\label{cap:metodologia}

%TODO revisar
% Este capítulo apresenta a metodologia adotada neste trabalho. A seção \ref{cap:metodologia:sec:conjunto_dados:sec:escolha_conjunto} descreve o processo de escolha de hotéis; a seção \ref{cap:metodologia:sec:conjunto_dados:sec:coleta_review} descreve como foi realizada a coleta das avaliações, \ref{cap:metodologia:sec:conjunto_dados:sec:pre_processamento} descreve o preparo e escolha das avaliações pós coleta.
% ; a seção \ref{subsec:pre_processamento} descreve // TODO; a seção \ref{subsec:analise_sentimentos} descreve // TODO; a seção \ref{subsec:analise_temporal} descreve // TODO

\section{Conjunto de Dados}
\label{cap:metodologia:sec:conjunto_dados}

%\subsection{Escolha do conjunto}
%\label{cap:metodologia:sec:conjunto_dados:sec:escolha_conjunto}

O conjunto de dados foi selecionado considerando os hotéis localizados em todo o território brasileiro e com informações disponíveis no \textit{Google Maps}.

% https://developers.google.com/maps/apis-by-platform?hl=pt-br#web_service_apis
A primeira etapa consiste em identificar os hotéis distribuídos no território brasileiro por estado na plataforma do \textit{Google Maps}, essa tarefa é automatizada por meio da utilização do serviço exposto do \textit{Google Maps}, a iteração foi realizada utilizando a biblioteca oficial do \textit{Google Maps} versão escrita em Java que está disponível publicamente no repositório do \citeonline{javaGoogleMapsService2022}.

Para identificar os diversos hotéis, foi escrito um \emph{script} Kotlin \cite{scriptKotlinBuscarHoteis} que interage com a biblioteca supracitada para realizar a busca da lista de hotéis de forma automatizada, realizando buscas de forma sequencial considerando todos os estados brasileiros e persistindo os dados obtidos.

O objetivo nesta etapa é identificar o maior número possível de hotéis que estão disponíveis para consulta na plataforma e em posse destes obter informações mais detalhadas, como quantidade de avaliações, nota atual, localização do estabelecimento, para cada hotel para ser possível em posse dessa lista então realizar escolha dos hotéis para estudo.

Para essa tarefa utilizaremos a interface de \emph{Text Search Query}, informações como o nome do hotel, localização e identificador do hotel na plataforma são disponibilizadas por essa interface. Nessa interface conseguimos realizar a busca passando informações que servirão para limitar a busca, como idioma de interesse, coordenadas e diâmetro limite para ser considerado e dessa forma obter uma busca com viés dos parâmetros fornecidos \cite{placesSearchText2023}.

\lstinputlisting[language=java,caption=Fragmento de código da interface \textit{Google Maps} \emph{Text Search Query}]{extras/code/googlemaps-place-search.kt}

O texto utilizado para a busca dos hotéis é definida no seguinte modelo “frase + região”. Por exemplo, na frase “hotéis próximos ao Cristo Redentor”, a API deverá retornar informações básicas de hotéis cadastrados na plataforma próximos à região do Cristo Redentor. As regiões utilizadas foram os estados e capitais brasileiros.

Se necessário serão adicionados de forma manual hotéis em regiões brasileiras populares por suas atrações turísticas, indicados na lista de regiões das principais atrações do Brasil segundo~\citeonline{googleFlights2022destinos}. Essa lista é composta com base nas análises conduzidas pelo Google, que considera a quantidade de menções online e as suas avaliações recebidas na plataforma..

Após obter a lista com as informações básicas se faz necessário obter informações mais detalhadas dos estabelecimentos, para isso será utilizada a interface de \emph{Place Details} para poder obter essas informações na plataforma, dentre as informações expostas por esta interface apenas algumas são de nosso interesse e serão utilizadas como, por exemplo, mas não se restringindo à, número de avaliações disponíveis, classificação atual na plataforma e quantidade de avaliações, essas serão informações importantes no momento de escolha dos hotéis que serão utilizados no estudo, essas e outras são obtidas nessa etapa. Infelizmente por limitação da plataforma essa interface apenas lista 5 avaliações aleatórias, tornado inviável a sua utilização para coleta de grande número de avaliações.

\lstinputlisting[language=java,caption=Fragmento de código da interface \textit{Google Maps} \emph{Place Details}]{extras/code/googlemaps-place-details.kt}

Por conta da grande variedade de hotéis, o critério de escolha será composto por limitar a hotéis com disponibilidade de pacote \emph{All Inclusive} e na região nordeste. Posteriormente essa lista irá servir como entrada para um \emph{Web Scraping} para buscar as avaliações onlines disponíveis no \textit{Google Maps}, realizadas pelos usuários e os \emph{Local Guides}~\cite{google2022localguides} registrados na plataforma. Os detalhes da escolha das avaliações são discutidos em \ref{cap:metodologia:sec:conjunto_dados:sec:pre_processamento}.

\subsection{Coleta de dados}
\label{cap:metodologia:sec:conjunto_dados:sec:coleta_review}

Já existe um \emph{script} escrito em Python \cite{gaspa93scrapper2023} que dado uma \emph{URL} do \textit{Google Maps}, utilizando Selenium, obtém um determinado número de avaliações, o \emph{script} porém estava defasado e pendente de atualização, um dos \emph{forks} existentes \citeonline{ryuuzakescrapper2023} estava também pouco mais atualizado, porém ainda não funcional.

Utilizando esses projetos como base fiz um novo \emph{fork} e então atualizei o \emph{script} para realizar o trabalho de obter as avaliações dos hotéis, o \emph{script} escrito em Python e as suas principais dependências eram de Selenium~\cite{selenium2023}, NumPy~\cite{harris2020array}, Pandas\cite{jeffreback20226702671} e Beautiful soup~\cite{richardson2007beautiful}. Utilizando docker~\cite{merkel2014docker} para execução do \emph{script} de forma concorrente, utilizando o Selenium e executando alguns \textit{drivers} do Google Chrome, porém para esse tipo de tarefa o Selenium não é uma ferramenta eficiente e seu desempenho não é rápido, satisfatório e nem necessário, existem diferentes impactos por conta de sua utilização, os quais não serão discutidos aqui, a busca das avaliações para apenas um hotel demorava aproximadamente 3h para executar e o sucesso da execução do \emph{script} com a utilização do Selenium depende diretamente da imutabilidade da interface visual do \textit{Google Maps}, o que torna esse formato não escalável dado a velocidade com que a interface do \textit{Google Maps} recebe atualizações. Durante a adaptação do \emph{script} para conseguir extrair corretamente as avaliações foram realizadas 10 tentativas de execução, no período de 09/2022 à 11/2022, com periodicidade aproximadamente semanal e aproximadamente a cada duas semanas a estrutura do \textit{Google Maps} mudava, estrutura essa que é parte fundamental para que a execução obtivesse sucesso, e dessa forma o \emph{script} precisava passar por nova atualização para que a nova estrutura fosse de conhecimento, existiam inclusive alguns testes no formato A/B onde alguns estabelecimentos possuíam estrutura da página do \textit{Google Maps} diferente de outros, aumentando ainda mais a complexidade de manter o \emph{script} funcional.

O \emph{script} possui uma dependência muito forte com a parte estrutural da página e dessa forma qualquer mudança em \emph{tags HTML} e/ou as classes \emph{CSS} da página dos estabelecimentos no \textit{Google Maps}, características de um \emph{WebScrapper}, que dificulta a manutenção e limita a sua capacidade de reutilização.

% https://github.com/emerson-matos/maps-reviews-api-scraper
Baseado no \emph{script} do \emph{WebScrapper} existe um repositório que utiliza diretamente uma API do Google chamada \emph{reviewDialog}, sem a necessidade de utilizar o Selenium, que era na maioria responsável pelo desempenho insatisfatorio, dessa forma a coleta das avaliações ocorre de forma bem mais rápida.
% adicionar isso? TODO
%, e em um período de aproximadamente 1h foi possível obter um total de 86.291 avaliações dos hotéis da lista.


\subsection{Pré-processamento dos dados}
\label{cap:metodologia:sec:conjunto_dados:sec:pre_processamento}

Após capturar as avaliações precisamos realizar alguns filtros e tratamentos para que a análise de sentimentos consiga resultar algo concreto, dessa forma foram definidos critérios de filtro para ter uma lista final com as avaliações a serem levadas em consideração na análise. Dessa forma definiremos alguns critérios para realização de filtro.

Os critérios são a avaliação precisa possuir conteúdo textual e este por sua vez precisa então possuir 3 ou mais caracteres, não ser traduzido e será realizado filtro de período, que será definido posteriormente.

\begin{table}[]
	\centering
	\begin{tabular}{|l|p{5cm}|p{5cm}|}
		\hline
		\textbf{Campo}            & \textbf{Descrição}                                                              & \textbf{Tipo}                                                                  \\
		\hline
		token                     & \emph{token} utilizado para buscar a próxima sequência de avaliações na API     & textual                                                                        \\
		\hline
		review\_id                & identificador da avaliação                                                      & textual                                                                        \\
		\hline
		retrieval\_date           & data na qual o \emph{script} recuperou a avaliação                              & textual em formato de datetime                                                 \\
		\hline
		rating                    & valor numérico de nota que foi atribuído ao estabelecimento                     & numérico                                                                       \\
		\hline
		rating\_max               & valor numérico máximo de nota que poderia ter sido atribuído ao estabelecimento & numérico                                                                       \\
		\hline
		relative\_date            & data relativa ao retrieval\_date de quando a avaliação foi publicada            & textual formato de data relativa(1 dia atrás, 1 ano atrás, 3 meses atrás, etc) \\
		\hline
		likes                     & quantidade de Likes que a avaliação recebeu de outros usuários da plataforma    & numérico                                                                       \\
		\hline
		other\_ratings            & avaliações de outros aspectos relacionados ao estabelecimento                   & textual                                                                        \\
		\hline
		trip\_type\_travel\_group & indicador se a viagem foi realizada em grupo                                    & textual                                                                        \\
		\hline
		user\_name                & nome do usuário que realizou a publicação                                       & textual                                                                        \\
		\hline
		user\_is\_local\_guide    & identificador para indicar usuários que são local guides                        & textual                                                                        \\
		\hline
		user\_reviews             & número para indicar a quantidade de avaliações realizadas pelo usuário          & textual                                                                        \\
		\hline
		user\_photos              & número de fotos que o usuário postou                                            & numérico                                                                       \\
		\hline
		user\_url                 & endereço na web do perfil do usuário                                            & textual                                                                        \\
		\hline
		text                      & conteúdo textual da avaliação                                                   & textual                                                                        \\
		\hline
		response\_text            & conteúdo textual de uma possível resposta do estabelecimento à avaliação        & textual                                                                        \\
		\hline
		response\_relative\_date  & data relativa ao retrieval\_date de quando a resposta à avaliação foi publicada & string no formato de data relativa                                             \\
		\hline
		errors                    & lista de erros ao tentar fazer o \emph{parse} da avaliação                      & lista textual                                                                  \\
		\hline
	\end{tabular}
	\caption{Estrutura da avaliação}
	\label{tab:estrutura_review}
\end{table}

Cada uma das avaliações é composta por 18 campos com informações com formatos diferentes e descritos na tabela \ref{tab:estrutura_review}.

Dentre os 18 campos, estaremos interessados apenas em alguns deles, sendo:

\begin{itemize}
	\item text: conteúdo que passará pela analise de sentimento
	\item rating: utilizado na analise de distribuição de notas.
	\item relative\_date: utilizado para inferir a o período no qual a avaliação foi submetida
	\item retrieval\_date: útil para que possa ser possível inferir a data na qual a avaliação foi submetida.
\end{itemize}

E adicionaremos algumas novas colunas para:

\begin{itemize}
	\item source: indica o hotel ao qual a avaliação foi submetida. exemplo: 'hotel-marsol-beach-resort'
	\item mes\_avaliacao: indica em qual mês a avaliação foi submetida. exemplo: 1, 6, 8.
	\item ano\_avaliacao: indica em qual ano a avaliação foi submetida. exemplo: 2023, 2021, 2016.
	\item mes\_ano\_avaliacao: indica em qual mes-ano a avaliação foi submetida, exemplo: '2023-01', '2016-07'
\end{itemize}

Para construir as colunas de tempo (mes\_avaliacao, ano\_avaliacao e mes\_ano\_avaliacao) é necessário realizar um tratamento de data relativa de quando a avaliação foi submetida, que é representado por relative\_date, com o retrieval\_date que representa o momento em que a avaliação foi obtida, o seguinte trecho de código foi utilizado para realizar o tratamento:

\lstinputlisting[language=Python,caption=Código utilizado para tratar a data relativa]{extras/code/from_relative_date_to_timestamp.py}

\section{Análise dos Dados}
\label{cap:metodologia:sec:analise_dados}

A análise de sentimentos foi realizada utilizando modelos de PLN, incluindo \textit{BERTs} pré-treinados e foi também explorada a utilização de LLMs (\textit{GPT} 3.5, Vicuna e OpenChat) para classificar, porém de forma genérica e similar ao formato que modelos \textit{BERTs} realizam.

Os modelos foram pré treinados em conjuntos de dados rotulados para reconhecer padrões de linguagem que indicam sentimentos positivos, negativos ou neutros. O objetivo da utilização de modelos diferentes para analisar todo o conjuro de avaliações obtidas para que então seja atribuíam uma etiqueta com base na probabilidade calculada para cada sentimento. Por exemplo, se a probabilidade de um sentimento positivo fosse maior que a de sentimentos negativos ou neutros, a etiqueta "positivo" será então atribuída à avaliação e utilizada como classificação final, os modelos conversacionais utilizados terão a tarefa de simplesmente retornar uma etiqueta numérica, sem necessidade de retornar a probabilidade de tal ou de realizar tarefa mais robusta de avaliação.

Embora essa abordagem simplificada não leve em consideração todos os aspectos sutis da linguagem humana, ela proporciona uma compreensão geral dos sentimentos expressos pelos usuários. Essas etiquetas podem ser úteis para avaliações rápidas e gerais de sentimentos em um grande volume de dados, permitindo uma análise inicial e rápida das tendências gerais de satisfação dos clientes em relação aos hotéis.

Para a análise de sentimentos, os textos das avaliações foram alimentados em modelos baseados em \textit{BERT} já pre treinados, processados utilizando a biblioteca \textit{Transformers} em \textit{python}. Esses modelos foram pre treinados em diferentes conjuntos de dados rotulados, explicitados em suas respectivas páginas referências, e utilizados para classificar o sentimento das avaliações em categorias como positivo, negativo ou neutro.

Foram selecionados 5 modelos \textit{BERT} para serem utilizados, escolhidos com base no ranking disponibilizado pelo \textit{hugging face}, sendo eles:

\begin{itemize}
	\item philschmid/distilbert-base-multilingual-cased-sentiment \cite{Sanh2019DistilBERTAD}
	\item lxyuan/distilbert-base-multilingual-cased-sentiments-student \cite{lik_xun_yuan_2023}
	\item citizenlab/twitter-xlm-roberta-base-sentiment-finetunned \cite{robertaCitizenlab2022}
	\item cardiffnlp/twitter-xlm-roberta-base-sentiment \cite{barbieri2022xlmtmultilinguallanguagemodels}
	\item ramonmedeiro1/bertimbau-products-reviews-pt-br \cite{bertimbauRamon2023}
\end{itemize}

Nessa etapa cada um dos modelos~\textit{BERTs} atribui ao conteúdo textual da avaliação uma pontuação indicando a confiança de uma dada etiqueta ser a classificação correta daquela avaliação. Após todas as analises terminarem foi realizado então a união dos resultados e a etiqueta escolhida será igual a do modelo \textit{BERT} que atribuir o \emph{score} mais alto.

Todo o código foi executado utilizando a biblioteca \textit{transformers}~\cite{Wolf_Transformers_State-of-the-Art_Natural_2020}, utilizando o \textit{AutoTokenizer}, \textit{AutoModelForSequenceClassification} e \textit{pipeline}, executados usando dispositivo Cuda (Nvidia RTX 4070TI com 12 GB de VRAM) %% mais detalhes do ambiente de execução?

Além dos modelos baseados em \textit{BERT}, os LLMs foram utilizados para ser possível entender as capacidades desses modelos em relação à simples tarefa de analise de sentimento das avaliações de hotéis e sua eficiência em relação aos \textit{BERTs}. Esses modelos foram utilizados para gerar respostas, objetivas assim como o \textit{BERT} para ser possível realizar a comparação dos resultados de forma simples.

Após a análise de sentimentos, utilizaremos os resultados agrupados por períodos de modo a conseguir visualizar em modo gráfico as avaliações divididas conforme explicitado na seção anterior para um grupo de hotéis.

Os resultados da análise de sentimentos foram avaliados em termos da capacidade de capturar com precisão o sentimento dos usuários nas avaliações de hotéis e tempo de processamento do corpus, comparando os diferentes modelos utilizados.
