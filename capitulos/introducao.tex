% ----------------------------------------------------------
% Introdução 
% Capítulo sem numeração, mas presente no Sumário
% ----------------------------------------------------------

\chapter[Introdução]{Introdução}
\label{cap:intro}
% \addcontentsline{toc}{chapter}{Introdução}


Com os avanços da tecnologia, o acesso à informação tem sido cada vez mais facilitado entre a população geral. Antes, as fontes de informações mais frequentes eram o conhecimento de prévio de indivíduos mais experientes ou conteúdos impressos \cite{jayathilake2021critical}. Uma pesquisa feita pela \citeonline{cetic2019pesquisa} indica que 3 entre cada 4 brasileiros possuem acesso à Internet e esses entrevistados também responderam que realizaram algum tipo de consulta online, sendo que 28\% procuraram por informações referentes a viagens e acomodações, um índice baixo se comparado aos 57\% que responderam ter buscado por informações sobre produtos e serviços. O acesso pelo meio digital tende a aumentar quando o foco é a população da Geração Z, sendo a geração que tende a possuir uma maior facilidade com a tecnologia \cite{sfodera2022technology}.

Quando surge o interesse em realizar viagens, sites como Booking.com, Tripadvisor e AirBnb, são algumas das referências em relação à reserva de hospedagem em hotel~\citeonline{best2022HostSites}. É feita uma busca por informações disponibilizadas por pessoas que já frequentaram os locais, sendo que cada usuário da plataforma preenche as avaliações conforme o modelo disponível, partindo de uma avaliação objetiva dividida por categorias, em que o usuário é questionado por área de interesse de modo a atribuir um valor numérico como nota para aquela experiência. Em muitos casos, também há um bloco de texto disponível para relatar a experiência em sua estadia. As plataformas citadas costumam ter, além das avaliações dos usuários, informações próprias, tais como preços por diárias e distâncias de atrações.

As informações disponíveis nessas plataformas costumam direcionar o usuário e o auxilia em sua tomada de decisão. Avaliações online são a segunda fonte mais confiável de informações, logo atrás das recomendações de familiares e amigos~\cite{chatterjee2020drivers}. Porém, qual a qualidade dessas informações? Essas avaliações variam durante o tempo? Um hotel está tendo avaliações melhores agora do que no mesmo período no ano passado? Esse local está sendo avaliado negativamente por conta de algum evento atípico? Mesmo registrando uma avaliação ruim, o usuário que a submeteu possui outra avaliação registrada no passado?
Além disso, as plataformas nem sempre expõem as informações de forma clara e objetiva, dificultando a tomada de decisões para o usuário, que possui o objetivo de escolher um local que contemple suas necessidades durante seu período de viagem. Afinal, o usuário não tem interesse em ser surpreendido por avaliações equivocadas ou que as informações mais recentes tenham sido ocultadas devido às tendências negativas por conta da organização por nível de popularidade nas plataformas.

Atualmente, o Google possui uma ferramenta de busca muito popular, presente nos mais diversos aparelhos celulares, e a frase "restaurantes próximos a mim" é a frase mais utilizada para busca de locais nos últimos cinco anos~\cite{li20213}. No Google Maps também há uma funcionalidade de avaliação de locais~\cite{googleMaps2022Blog}, conforme descrito no próprio blog deles:

\begin{quote}
    Para quem quer conhecer lugares novos, as avaliações deixadas por usuários do Google são uma rica fonte de conhecimento e informações locais. Elas ajudam a encontrar estabelecimentos e empresas que se encaixam no gosto de cada um — seja uma padaria com o melhor bolo de laranja sem glúten da cidade ou um restaurante de bairro com deliciosa música ao vivo.
\end{quote}

As avaliações do Google Maps podem ser escritas por qualquer usuário da ferramenta, o que pode ser um ponto negativo se for considerado que algumas das plataformas supracitadas exigem que o internauta necessariamente tenha se hospedado no hotel que deseja avaliar.

\section{Justificativa}

Vivemos em um mundo com muitas informações disponíveis e que são de fácil acesso. Porém, está cada vez mais difícil interpretar e filtrar as informações que irão agregar maior conhecimento de modo a enriquecê-lo sobre um determinado assunto.

No cenário em que procuramos informações sobre possíveis estadias é importante saber selecionar as avaliações online com as informações mais próximas da realidade atual do estabelecimento. Neste projeto, queremos diminuir o viés intrínseco contido individualmente nas avaliações de modo a obter a informação necessária para conseguir selecionar uma estadia que irá fornecer o maior valor para o usuário final.

\section{Objetivos}

\subsection{Geral}

O objetivo geral deste projeto é implementar um método para análise de sentimentos temporal de avaliações online de hotéis disponíveis no Google Maps. Com isso, espera-se facilitar o entendimento dessas avaliações para os usuários que estiverem procurando informações de hospedagem, explicitando como as avaliações variam ao longo do tempo.

\subsection{Específicos}

\begin{itemize}
    \item Criar um corpus com as avaliações online de hotéis disponíveis no Google Maps.
    \item Interpretar o sentimento dessas avaliações individualmente.
    \item Analisar a variação do sentimento das avaliações dos hotéis ao longo do tempo.
\end{itemize}
