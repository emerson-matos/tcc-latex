\chapter{Cronograma}
\label{cap:cronograma}

% \begin{figure}[h]
    % \center
    % \includegraphics[width=17cm]{figs/cronograma.png}
% \end{figure}

Inicialmente, será criado um corpus para análise com conteúdo suficiente para obter variação significativa nos sentimentos com o avanço do tempo, para ser possível identificar visualmente essas variações por hotel, tópico e por região.

Após essa etapa será necessário realizar o tratamento dos dados e o treinamento do modelo BERT para realizar a classificação do corpus.

Com o treinamento concluído, espera-se realizar o emparelhamento de adjetivo-substantivo e a avaliação de sentimento deles e de todo o corpus.

Com os resultados em mãos, será realizada um agrupamento por região e hotéis, visando mostrar de forma gráfica como essas avaliações têm variado temporalmente.

De forma resumida, o seguinte cronograma de atividades deverá ser considerado:

\begin{table}[H]
    \centering
    {\footnotesize
    \begin{tabular}{|p{6.1cm}|l|l|l|l|l|l|l|l|l|l|l|l|}
    \hline
    \multirow{2}{*}{ATIVIDADES} & \multicolumn{5}{c|}{PGC II} & \multicolumn{4}{c|}{PGC III}\\ \cline{2-10}
                                           &Set&Out&Nov&Dez&Jan&Jun&Jul&Ago&Set \\ \hline
    1. Conjunto de Dados                   & X & X & X &   &   &   &   &   &    \\ \hline
    2. Pré-processamento dos Dados         &   &   & X & X & X &   &   &   &    \\ \hline
    3. Análise de Sentimentos              &   &   &   &   &   & X & X &   &    \\ \hline
    4. Análise Temporal dos Dados          &   &   &   &   &   & X & X &   &    \\ \hline
    5. Escrita de Relatórios               &   &   &   & X & X &   & X & X &    \\ \hline
    6. Defesa                              &   &   &   &   &   &   &   &   & X  \\ \hline
    \end{tabular}}
    \caption{Cronograma de Atividades}
    \label{tab:cronograma}
\end{table}
