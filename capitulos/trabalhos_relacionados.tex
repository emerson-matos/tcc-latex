\chapter{Trabalhos Relacionados}
\label{cap:trabalhos_relacionados}

% todo

\section{Análise em larga escala da evolução temporal de tópicos obtidos do twitter basado em Apache Spark}
\label{cap:trabalhos_relacionados:sec:braulio}

No trabalho de \citeonline{vinces2022}, foi realizada uma análise abrangente da evolução temporal dos tópicos extraídos do \textit{Twitter}, empregando o \textit{Apache Spark} para processamento distribuído. A modelagem de tópicos em redes sociais, como o \textit{Twitter}, desempenha um papel crucial em diversas aplicações, como detecção de notícias e análise de sentimentos. O objetivo principal foi obter resultados comparáveis ou superiores ao modelo \textit{Latent Dirichlet Allocation} (LDA), com foco na escalabilidade e eficiência computacional, utilizando modelos \textit{BERT} para aprimorar a extração de características textuais.

Durante a pesquisa, foram explorados conceitos teóricos relacionados à modelagem de tópicos, incluindo \textit{Word Embeddings} e métricas de avaliação, com ênfase na análise temporal dos tópicos identificados. A metodologia adotada envolveu um pipeline de trabalho estruturado em cinco etapas, visando uma abordagem eficaz para a análise dos dados do \textit{Twitter} ao longo do tempo. Os resultados experimentais, impulsionados pela utilização de modelos \textit{BERT} pré-treinados, foram fundamentais para as conclusões alcançadas, destacando a relevância e o impacto do estudo no campo da Inteligência Artificial e textit{Big Data}.


\section{Análise do atencimento bancário relacionado a espólio na plataforma consumidor.gov.br}
\label{cap:trabalhos_relacionados:sec:desterro}

O trabalho de \citeonline{desterro2023} teve como objetivo analisar reclamações e avaliações relacionadas a espólios registradas contra instituições bancárias brasileiras na plataforma \textit{Consumidor.gov.br}, focando especialmente no período da pandemia de COVID-19. Para isso, foram utilizadas técnicas de Processamento de Linguagem Natural (PLN) para analisar os sentimentos das avaliações dos consumidores e identificar os principais tópicos das reclamações. Os dados foram extraídos por meio de \textit{web scraping}, empregando o modelo \textit{XLM-T} para a análise de sentimentos e o algoritmo de clusterização \textit{K-Means} para identificar os tópicos predominantes.

A pesquisa revelou que os representantes de clientes falecidos enfrentam principalmente problemas relacionados ao encerramento de contas, documentação para inventário e atendimento nas agências bancárias. Além disso, foi observada uma predominância de sentimentos negativos nas avaliações dos consumidores. Esses resultados são importantes pois podem ajudar as instituições bancárias a identificar e resolver problemas recorrentes no atendimento aos representantes de clientes falecidos, melhorando assim a qualidade do serviço prestado.

O estudo de Desterro destaca a utilidade de aplicar técnicas de PLN na análise de opinião de consumidores, demonstrando que essas técnicas podem ser valiosas para empresas que buscam aprimorar seu atendimento ao cliente. A análise detalhada e sistemática das reclamações pode fornecer  percepções importantes para a implementação de melhorias nos processos e na comunicação com os clientes, especialmente em situações sensíveis como o gerenciamento de espólios.
