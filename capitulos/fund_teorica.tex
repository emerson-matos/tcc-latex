\chapter{Fundamentação Teórica}
\label{cap:fund_teorica}

Neste capítulo abordaremos conceitos, necessários para o desenvolvimento e entendimento do projeto, do campo de estudo de processamento de linguagem natural, mais especificamente a análise de sentimentos. Também iremos discorrer sobre a funcionalidade de avaliação de estabelecimentos exposta pelo Google Maps.

\section{Nuvem de Palavras}
\label{cap:fund_teorica:sec:word_cloud}
% TODO

\section{Análise de Sentimento}
\label{cap:fund_teorica:sec:analise_sentimento}
% 10.1145/1014052.1014073 Mining and Summarizing Customer Reviews
A análise de sentimentos é uma das áreas de pesquisa em processamento de linguagem natural, que visa utilizar recursos computacionais para estudar os sentimentos, opiniões e emoções de uma determinada sentença em relação a um alvo \cite{Bellini2019}.

Uma opinião é composta de dois componentes chaves, formando o par \emph{(g,s)}: um alvo \emph{g} e um sentimento \emph{s} relacionado a esse alvo. E como destacado em \cite{Machado2018}, \emph{g} pode se referir a qualquer aspecto da entidade ou a ela própria e sentimento \emph{s} pode ser positivo, negativo ou neutro, ou ser representado por um valor numérico que representa a sua intensidade, denominado como polarização do sentimento ou opinião.



\section{modelos}
\label{cap:fund_teorica:sec:modelos}
% TODO bert
% % https://arxiv.org/pdf/1909.10649.pdf
% TODO LLM
% https://arxiv.org/pdf/2307.06435.pdf

\section{Avaliação de estabelecimentos}
\label{cap:fund_teorica:sec:avaliacao_estabelecimentos}
% falar sobre o google maps e as avaliações e que avaliações de idioma diferente são traduzidas. traduzido por google
% preprocessamento
% pesquisar sobre os algoritmos de analise de sentimento https://www.lume.ufrgs.br/bitstream/handle/10183/140910/000991520.pdf?sequence=1
% https://www.teses.usp.br/teses/disponiveis/59/59143/tde-31102018-174346/publico/mateuscorrigida.pdf
% pesquisar sobre classificação de avaliações
% topicos? https://sol.sbc.org.br/livros/index.php/sbc/catalog/download/4/177/354-1?inline=1
% hugging face?
% todo
