\chapter{Fundamentação Teórica}
\label{cap:fund_teorica}


%Neste capítulo abordaremos alguns dos conceitos necessários para o desenvolvimento e entendimento do projeto, do campo de estudo de processamento de linguagem natural, mais especificamente a análise de sentimentos. Também iremos discorrer sobre a funcionalidade de avaliação de estabelecimentos exposta pelo Google Maps.

\section{Avaliações no Google Maps}
\label{cap:fund_teorica:sec:google_maps}

O Google Maps \citeonline{wiki:googlemaps2023} é uma das ferramentas de mapeamento e navegação mais populares do mundo, desenvolvida e mantida pela Google. Lançado em 2005, o Google Maps rapidamente se tornou uma ferramenta essencial para milhões de pessoas ao redor do mundo, oferecendo mapas detalhados, direções de navegação, informações sobre o tráfego em tempo real e uma ampla gama de recursos adicionais, como visualização de imagens de satélite, avaliações de estabelecimentos comerciais e muito mais.

Um dos recursos interessantes do Google Maps é o programa \emph{"Local Guides"}~\citeonline{google2022localguides}. Este programa foi introduzido pela Google para incentivar os usuários a contribuir com informações adicionais sobre lugares e estabelecimentos em suas comunidades locais. Os \emph{"Local Guides"}~são usuários voluntários que compartilham avaliações, fotos, informações sobre horários de funcionamento, preços e outros detalhes úteis sobre restaurantes, lojas, atrações turísticas e outros pontos de interesse.

Os \emph{"Local Guides"}~são incentivados pela Google a contribuir com conteúdo de qualidade, e em troca, eles podem ganhar pontos, níveis e até mesmo recompensas, como armazenamento adicional no Google Drive, acesso antecipado a novos recursos do Google Maps e até mesmo convites para eventos exclusivos. Essa gamificação do processo de contribuição ajuda a manter os usuários engajados e motivados a compartilhar informações úteis.

Além disso, os \emph{"Local Guides"}~desempenham um papel importante na melhoria contínua da precisão e da utilidade do Google Maps. Ao fornecer informações atualizadas e detalhadas sobre os locais, eles ajudam outros usuários a tomar decisões informadas sobre onde ir e o que fazer. Essa contribuição colaborativa cria uma comunidade global de usuários que trabalham juntos para tornar o Google Maps uma ferramenta ainda mais poderosa e abrangente.

Em resumo, o Google Maps é uma ferramenta essencial para navegação e descoberta de lugares, enquanto os \emph{"Local Guides"}~são membros voluntários da comunidade que contribuem com informações valiosas para melhorar a precisão e utilidade do serviço. Juntos, eles desempenham um papel crucial na criação de uma experiência de mapeamento e navegação mais rica e envolvente para todos os usuários.

%\subsection{Avaliação de estabelecimentos}
%\label{cap:fund_teorica:sec:google_maps:subsec:avaliacao_estabelecimentos}


A funcionalidade de avaliação de estabelecimentos no Google Maps é uma ferramenta poderosa que oferece aos usuários uma maneira de compartilhar suas experiências e opiniões sobre diferentes locais, desde restaurantes e cafés até hotéis e pontos turísticos. Essa funcionalidade permite que os usuários forneçam feedback valioso sobre a qualidade do serviço, a atmosfera do local, a qualidade dos produtos e uma infinidade de outros aspectos.

Uma das principais vantagens dessa funcionalidade é a capacidade de ajudar outros usuários a tomar decisões informadas ao escolherem um estabelecimento para visitar. As avaliações e comentários fornecidos por outros usuários podem servir como uma referência confiável, permitindo que as pessoas tenham uma noção melhor do que esperar de um determinado local antes de decidirem visitá-lo.

Além disso, o sistema de avaliação do Google Maps geralmente é muito acessível e fácil de usar. Os usuários podem rapidamente atribuir uma classificação de estrelas e deixar um comentário detalhado sobre sua experiência com apenas alguns cliques, ajudando a manter uma base de dados robusta e atualizada sobre os estabelecimentos locais.

No entanto, é importante reconhecer que nem todas as avaliações podem ser completamente imparciais ou precisas. Algumas podem ser tendenciosas ou baseadas em experiências individuais que podem não refletir a experiência média dos clientes. Portanto, é sempre aconselhável considerar várias avaliações e fontes de informações ao tomar uma decisão.

Em resumo, a funcionalidade de avaliação de estabelecimentos no Google Maps oferece aos usuários uma maneira conveniente de compartilhar e acessar informações sobre locais locais, ajudando a tornar as experiências de viagem e exploração mais informadas e satisfatórias.
% falar sobre o google maps e as avaliações e que avaliações de idioma diferente são traduzidas. traduzido por google
% preprocessamento
% pesquisar sobre os algoritmos de analise de sentimento https://www.lume.ufrgs.br/bitstream/handle/10183/140910/000991520.pdf?sequence=1
% https://www.teses.usp.br/teses/disponiveis/59/59143/tde-31102018-174346/publico/mateuscorrigida.pdf
% pesquisar sobre classificação de avaliações

\section{Processamento de Linguagem Natural}



\subsection{Grandes Modelos de Linguagem}
% hugging face?
% todo


%\subsection[Large Language Models]{LLM}
%\label{cap:fund_teorica:sec:modelos:subsec:LLM}
% TODO LLM
% https://arxiv.org/pdf/2307.06435.pdf
% \cite{naveed2024comprehensive}



Neste projeto foram utilizados Grandes Modelos de linguagem(\emph{Large Language Models} - LLM) pré-treinados, como o GPT (Generative Pre-trained Transformer), OpenChat e Vicuna, além de modelos baseados na arquitetura \emph{BERT} (Bidirectional Encoder Representations from Transformers)\cite{hugoZanini2021mediu}. Esses modelos representam o estado-da-arte em PLN e têm demonstrado um desempenho significativo em uma variedade de tarefas, incluindo análise de sentimentos.


\subsubsection[BERT]{BERT}
\label{cap:fund_teorica:sec:modelos:subsec:bert}

% TODO bert
% http://arxiv.org/abs/1909.10649

% % https://arxiv.org/pdf/1909.10649.pdf
Um modelo baseado em \emph{BERT}(\textit{Bidirectional Encoder Representations from Transformers}) é um tipo de modelo de linguagem pré-treinado desenvolvido pelo Google. \emph{BERT} é um modelo de aprendizado profundo que utiliza a arquitetura de transformers para capturar representações bidirecionais de palavras em um texto. Isso significa que ele é capaz de entender o contexto de uma palavra com base em suas palavras vizinhas, tanto anteriores quanto posteriores, em uma sentença.

Os modelos BERT são uma família de modelos de linguagem baseados em transformers que foram pré-treinados em grandes corpora de texto. Eles são capazes de capturar o contexto bidirecional das palavras em uma frase, o que os torna especialmente eficazes para tarefas de análise de sentimentos, onde o contexto é crucial para determinar o sentimento expresso.

O treinamento de \emph{BERT} é realizado em grandes quantidades de dados de texto, onde o modelo aprende a prever palavras em uma frase ou sentença com base no contexto global da sentença. Como resultado, \emph{BERT} é capaz de capturar nuances e complexidades da linguagem natural de uma maneira mais eficaz do que modelos mais simples.

Depois de pré-treinado em um grande corpus de texto, um modelo \emph{BERT} pode ser ajustado para tarefas específicas, como análise de sentimentos, classificação de texto, resposta a perguntas, entre outras, por meio de um processo chamado de ajuste fino (\textit{fine-tuning}). Durante o ajuste fino, o modelo é treinado em um conjunto de dados rotulado específico para a tarefa em questão, adaptando suas representações aprendidas durante o pré-treinamento para a tarefa específica.

Os modelos baseados em \emph{BERT} tornaram-se amplamente populares e são frequentemente usados como base para uma variedade de tarefas de processamento de linguagem natural devido à sua capacidade de capturar contextos complexos e produzir resultados de alta qualidade, BERTs treinados em português e com boa performance podemos citar o RoBERTa~\cite{liu2019roberta} e BERTimbau~\cite{souza2020bertimbau}.

\subsubsection{Vicuna}

O Vicuna \cite{vicuna2023} é um modelo de assistente de chat desenvolvido pela LMSYS, ajustado a partir do modelo LLaMA usando ajuste fino de instrução supervisionada. É um modelo de linguagem auto-regressivo baseado na arquitetura \emph{transformer}, projetado para tarefas de geração de texto. Os dados de treinamento para o Vicuna v1.3 consistem aproximadamente de 125.000 conversas coletadas do ShareGPT.com, enquanto o Vicuna v1.1 foi treinado com cerca de 70.000 conversas da mesma fonte.

Vicuna é avaliado usando \emph{benchmarks} padrão, preferência humana e LLM-como-juiz, fornecendo uma avaliação abrangente de seu desempenho. É notável por alcançar mais de 90\% da qualidade do ChatGPT em testes de preferência do usuário e superar significativamente o Alpaca, tornando-se um forte concorrente na família de modelos LLaMA ajustados por instrução até maio de 2023.

O modelo está disponível no \emph{Hugging Face}, onde tem sido utilizado em vários espaços e projetos, indicando sua ampla adoção e aplicação na comunidade de desenvolvedores. No entanto, é importante notar que o Vicuna está sujeito a uma licença não comercial, restringindo seu uso em aplicações comerciais.

Para aqueles interessados em começar com o Vicuna, o cartão de modelo do Hugging Face fornece informações detalhadas sobre detalhes do modelo, fontes, usos e como começar a trabalhar com o modelo. Além disso, há uma nota sobre uma versão mais recente dos pesos disponíveis, sugerindo que os usuários verifiquem atualizações para garantir que estejam usando a versão mais atual do modelo.

\subsubsection{GPT 3.5}

O GPT é um modelo baseado em transformadores que aprende representações de linguagem generalizadas a partir de grandes quantidades de texto não rotulado. Ele é capaz de gerar texto coerente e relevante, e pode ser adaptado para tarefas específicas, como análise de sentimentos, através do treinamento supervisionado em conjuntos de dados rotulados.

% https://en.wikipedia.org/wiki/GPT-3
O GPT-3.5, introduzido pela OpenAI em março de 2022, é uma subclasse dos modelos GPT-3 que representa um avanço significativo sobre seus predecessores. Ele é projetado para entender e gerar linguagem natural ou código, tornando-o adequado para uma ampla gama de aplicações, incluindo conversação e tarefas gerais. O GPT-3.5 é notável por sua velocidade e flexibilidade, sendo mais rápido que o GPT-4 e mais adaptável que o GPT Base. É considerado o modelo "suficientemente bom" para a maioria das necessidades, oferecendo um equilíbrio entre desempenho e custo-efetividade. A variante GPT-3.5 Turbo, em particular, é destacada como o melhor modelo da série GPT-3.5, usado pela versão gratuita do ChatGPT e elogiado por sua relação custo-eficácia e flexibilidade.

% https://platform.openai.com/docs/models/gpt-3-5-turbo
% https://openai.com/pricing 24/04/2024
Os modelos GPT-3.5, incluindo a variante Turbo, têm um limite máximo de token de 4.096 para a versão padrão e 16.384 para a versão Turbo-16k. Eles são treinados em dados até setembro de 2021, garantindo que tenham uma ampla base de conhecimento para se basear. A precificação para usar esses modelos é competitiva, com a variante Turbo custando \$0.0015 por 1.000 tokens para entrada e \$0.002 por 1.000 tokens para saída, enquanto a variante Turbo-16k custa \$0.0005 por 1.000 tokens para entrada e \$0.0015 por 1.000 tokens para saída.

\subsubsection{OpenChat}


O OpenChat \cite{wang2023openchat}  é um modelo de conversação baseado em transformadores que foi pré-treinado em uma grande quantidade de dados de conversação. Ele é capaz de gerar respostas contextuais e relevantes em conversas humanas, o que pode ser útil para entender o contexto em avaliações de hotéis e extrair sentimentos.


\subsection{Análise de Sentimentos}
%\label{cap:fund_teorica:sec:analise_sentimento}


%A análise de sentimentos é uma área da ciência da computação e do processamento de linguagem natural (PLN) que tem como objetivo identificar e extrair informações sobre sentimentos, opiniões e atitudes expressas em textos escritos.

% TODO
%
Uma técnica bastante popular é o agrupamento dos tokens para que seja posteriormente realizada uma representação visual dos dados, onde o tamanho do token representa sua frequência em um dado corpus, e sua coloração pode ser util para representar o sentimento do token. Essa representação gráfica é nomeada \emph{wordcloud}, ou nuvem de palavras.

O modelo de gráfico de nuvens de palavras é uma técnica de visualização frequentemente utilizada na análise de sentimentos para representar as palavras mais frequentes em um conjunto de dados de texto, com o tamanho das palavras proporcional à sua frequência de ocorrência. Neste contexto, as palavras são tipicamente coloridas de acordo com seu sentimento associado, o que facilita a identificação visual das palavras mais relevantes e das tendências sentimentais no texto.

Essa abordagem é valiosa na análise de sentimentos, pois permite uma rápida identificação de palavras-chave associadas a diferentes emoções, como positivas, negativas ou neutras. Ao visualizar um gráfico de nuvens de palavras, os analistas podem extrair insights sobre as percepções dos usuários e as principais tendências de sentimentos presentes nos comentários ou avaliações de um determinado produto, serviço ou tópico.

O modelo de gráfico de nuvens de palavras é uma ferramenta poderosa para a visualização e interpretação de dados de texto em análises de sentimentos, proporcionando uma representação intuitiva e acessível das informações linguísticas subjacentes.

% 10.1145/1014052.1014073 Mining and Summarizing Customer Reviews
A análise de sentimentos é uma das áreas de pesquisa em processamento de linguagem natural, que visa utilizar recursos computacionais para estudar os sentimentos, opiniões e emoções de uma determinada sentença em relação a um alvo \cite{Bellini2019}.

Uma opinião é composta de dois componentes chaves, formando o par \emph{(g,s)}: um alvo \emph{g} e um sentimento \emph{s} relacionado a esse alvo. E como destacado em \cite{Machado2018}, \emph{g} pode se referir a qualquer aspecto da entidade ou a ela própria e sentimento \emph{s} pode ser positivo, negativo ou neutro, ou ser representado por um valor numérico que representa a sua intensidade, denominado como polarização do sentimento ou opinião.


% topicos? https://sol.sbc.org.br/livros/index.php/sbc/catalog/download/4/177/354-1?inline=1


