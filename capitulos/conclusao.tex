\chapter{Conclusões}
\label{cap:conclusao}

\section{Contribuições}
\label{cap:conclusao:sec:contribuicoes}
% TODO

Este projeto apresentou uma análise abrangente de sentimentos temporais baseada em avaliações online de hotéis no \textit{Google Maps}, utilizando modelos de linguagem de última geração, como \textit{BERT}, \textit{Vicuna}, \textit{GPT 3.5} e \textit{OpenChat}. Através desta investigação, foi possível verificar que, para tarefas de classificação de sentimento a nível de documento, modelos mais simples e menos complexos como o \textit{BERT} são capazes de oferecer resultados satisfatórios, principalmente quando consideramos o custo-benefício em termos de tempo de execução e recursos computacionais.

Também fica evidente que não há necessidade de utilização de modelos proprietários, pois modelos não proprietários tem desempenho satisfatórios na tarefa de classificação de sentimento, porém com a necessidade de possuir uma infraestrutura adequada para que possa ser possível a execução dos mesmos.

Além disso, também fica evidente que a abordagem utilizada no projeto é suficientemente boa para ser utilizada em plataformas de recomendação de hospedagem, e podem inclusive ser aprimorada para fornecer mais detalhes sobre os pontos de destaque tanto positivos como negativos de um dado hotel, evitando a necessidade dos usuários de realizar de forma manual a investigação e revisão das avaliações, que em muitos cenários tomaria um tempo considerável para que seja realizado por humanos.

\section{Limitações}
\label{cap:conclusao:sec:limitacoes}

Uma das limitações desse trabalho é em relação aos recursos disponíveis para execução dos modelos abertos. Por conta do ambiente de execução utilizado possuir apenas 12 GB de VRAM não foi possível realizar testes com modelos maiores, e por isso não foi possível adicionar modelos maiores para comparação e como consequência não temos informação suficiente para podermos afirmar que modelos maiores sejam mais ou menos eficientes na execução da tarefa proposta. Outro grande impacto é em relação ao texto (\textit{prompt}) utilizado como início da conversação com cada modelo, em que alguns modelos podem ser mais sensíveis ao texto utilizado para inciar a tarefa de classificação do texto da avaliação, se faz necessário um estudo mais aprofundado para entender os possíveis impactos da utilização de diferentes textos com diferentes modelos.

Fica difícil notar variação de sentimento temporal com o conjunto de dados utilizado, principalmente por conta da característica de serem hotéis de grande qualidade e por esse motivo o volume de dados com sentimento positivo foi esmagadoramente superior aos outros. Outra forma de aprimorar seria realizar uma análise mais granular com relação a data que não foi possível por conta da limitação da plataforma do \textit{Google Maps}, ou utilizar um conjunto de hotéis de menor qualidade, em que é esperado que as avaliações com sentimento positivo sejam mais equivalentes, em termos de quantidade, com relação as outras.

\section{Trabalhos Futuros}
\label{cap:conclusao:sec:trab_futuros}

Para trabalhos futuros podemos explorar a análise de impacto do texto inicial utilizado para que o modelo realize a tarefa, e entender a sensibilidade dos modelos para com as alterações no conteúdo utilizado como estimulo para início do processamento, bem como o impacto da utilização de modelos iguais porém com tamanho de parâmetros diferentes. Além disso, pode ser explorado também a utilização de um corpus com uma variação maior de nota de classificação para tentar identificar uma variação de sentimento maior com relação ao tempo, ou até uma possível exploração com maior detalhe utilizando o poder de processamento dos grandes modelos de linguagem, fazendo com que a tarefa não seja uma simples atribuição de etiqueta de sentimento, mas também identificação do sentimento das entidades presentes nos documentos, isto é, o sentimento especifico para uma dada característica do hotel. Também poderia inverter e no lugar de olhar o conjunto de avaliações de um dado estabelecimento passar a olhar para o conjunto de avaliações de uma lista de \textit{local guides} e entender a relação de sentimento de suas avaliações e a nota atribuída e também a aplicação de técnicas mais avançadas de pré-processamento poderiam fornecer resultados ainda mais precisos e abrangentes.
