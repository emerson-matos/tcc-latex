% ----------------------------------------------------------
% Introdução 
% Capítulo sem numeração, mas presente no Sumário
% ----------------------------------------------------------

\chapter[Introdução]{Introdução}
\label{cap:intro}
% \addcontentsline{toc}{chapter}{Introdução}
% pesquisa https://agenciabrasil.ebc.com.br/geral/noticia/2020-05/brasil-tem-134-milhoes-de-usuarios-de-internet-aponta-pesquisa#:~:text=Os%20recursos%20mais%20utilizados%20s%C3%A3o,eletr%C3%B4nico%20(39%25)%20e%20participa%C3%A7%C3%A3o
% https://cetic.br/pt/tics/domicilios/2021/individuos/C6/
% https://books.google.com.br/books?hl=pt-BR&lr=&id=cHYqBF3G3lkC&oi=fnd&pg=PR7&dq=facilidade+no+acesso+a+informacao&ots=g9ggkZ8lS0&sig=-GlYLDNbFKs0q5WYfttQiXkphdI#v=onepage&q=facilidade%20no%20acesso%20a%20informacao&f=false
\begin{comment}
Neste capítulo precisamos:
\begin{itemize}
    \item Introduzir o contexto
    \item Definir o que entendemos como avaliação de hoteis
    \item Apresentar em linhas gerais quais métodos são usados para detectar fake news, qual o estada da arte atual e quais são suas limitações.
    \item Descrever a nossa proposta e objetivos
    \item Descrever a estrutura do relatório
\end{itemize}

\end{comment}

Com os avanços da tecnologia, o acesso a informação tem sido cada vez mais facilitado entre a população geral. Antes, as fontes de informações mais frequentes eram o conhecimento de previo de indivíduos mais experientes ou conteúdos impressos. Uma pesquisa feita pela CETIC em 2021 indicava que 3 entre cada 4 brasileiros possuem acesso a internet e esses entrevistados também responderam que realizaram algum tipo de consulta online, onde 28\% procurou por informações referentes a viagens e acomodações, um índice baixo se comparado aos 57\% que respondeu ter buscado por informações sobre produtos e serviços.

% https://blog.buscaonibus.com.br/melhores-sites-de-hospedagem-para-sua-viagem/

Quando surge o interesse em realizar viagens, sites como TripAdvisor, Trivago, Booking, AirBnb, são as principais fontes de informações para locais ainda desconhecidos. É feita uma busca de informações de pessoas que já frequentaram os locais, onde cada usuário da plataforma preenche as avaliações de acordo com o modelo disponível em cada plataforma, partindo de uma avaliação objetiva dividida por categorias, onde o usuário é questionado por área de interesse de modo a atribuir um valor número como nota para aquela experiência, em muitos casos, existindo um bloco de texto disponível para relatar a experiência em sua estadia local. As plataformas citadas costumam ter além das avaliações dos usuários, informações próprias em que a plataforma preenche, incluindo informações de preços por diárias e distâncias de atrações

Essas informações costumam direcionar o internauta e o auxiliam em sua tomada de decisão, porém qual a validade dessas informações? como essas avaliações variam durante o tempo? esse hotel está tendo avaliações melhores agora do que no mesmo periodo no ano passado? esse local esta sendo avaliado de forma negativa por conta de algum evento atipico? mesmo registrando uma avaliação ruim, o usuário que a submeteu possui outra avaliação registrada no passado?

Contudo, as plataformas nem sempre expõem as informações de forma clara e objetiva, dificultando a tomada de decisões para o internauta, que possui o objetivo de escolher um local que contemple suas necessidades durante seu período de viagem, afinal, o internauta não tem interesse em ser surpreendido que as avaliações dispostas estavam equivocadas e as informações mais recentes tenham sido ocultadas devido as tendências negativas por conta da organização por nivel de popularidade nas plataforma.

\begin{comment}
O presente relatório está estruturado da seguinte forma: o capítulo~\ref{cap:justificativa} apresenta..., o capítulo~\ref{cap:fund_teorica}... O capítulo~\ref{cap:metodologia} ..., o capítulo~\ref{cap:resultados} .... O capítulo~\ref{cap:conclusao} 

Demonstração de citação: o software de análise foi desenvolvido na linguagem Python~\cite{van1995python}, usando as bibliotecas Pandas~\cite{mckinney2010data} e Scikit-learn~\cite{scikit-learn}.
\end{comment}

\section{Objetivo}

O trabalho tem como objetivo realizar a análise de avaliações de hotéis disponibilizadas publicamente em plataformas online, afim de elaborar uma ferramenta que facilite o entendimento dessas avaliações para os internaltas que estiverem procurando informações de hospedagem, explicitando como as avaliações do recinto variam conforme o tempo.

\begin{comment}
    como o sentimento dos usuários que avaliaram o estabelecimento variou durante o tempo, se o recinto está recebendo avaliações com sentimentos mais positivos ou se a tendência é de que as avaliações continuem com sentimentos cada vez mais negativos, e identificiar possíveis mudanças de comportamento, que para esses cenários podem ser justificados por mudanças de equipe, mudanças de políticas internas da empresa ou por uma simples manutenção ou evolução das instalações
    , afim de observar e entender as variações de sentimentos das avaliações realizadas pelos usuários da plataforma distribuidos durante o tempo.
\end{comment}

\section{Justificativa}

Estamos vivendo em um mundo com muitas informações disponiveis e de fácil acesso, porém esta cada vez mais difícil filtrar e escolher as informações que vão fornecer mais insumo para o 