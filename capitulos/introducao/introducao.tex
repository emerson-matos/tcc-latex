% ----------------------------------------------------------
% Introdução 
% Capítulo sem numeração, mas presente no Sumário
% ----------------------------------------------------------

\chapter[Introdução]{Introdução}
\label{cap:intro}
% \addcontentsline{toc}{chapter}{Introdução}

JUSTIFICATIVA

Neste capítulo precisamos:
\begin{itemize}
    \item Introduzir o contexto
    \item Definir o que entendemos como avaliação de hoteis
    \item Apresentar em linhas gerais quais métodos são usados para detectar fake news, qual o estada da arte atual e quais são suas limitações.
    \item Descrever a nossa proposta e objetivos
    \item Descrever a estrutura do relatório
\end{itemize}

O presente relatório está estruturado da seguinte forma: o capítulo~\ref{cap:justificativa} apresenta..., o capítulo~\ref{cap:fund_teorica}... O capítulo~\ref{cap:metodologia} ..., o capítulo~\ref{cap:resultados} .... O capítulo~\ref{cap:conclusao} 

Demonstração de citação: o software de análise foi desenvolvido na linguagem Python~\cite{van1995python}, usando as bibliotecas Pandas~\cite{mckinney2010data} e Scikit-learn~\cite{scikit-learn}.

O presente trabalho tem como objetivo fazer analise de avaliações de hoteis que foram realizadas e disponibilizadas publicamente em plataformas online, utilizaremos como fonte de dados as avaliações disponiveis no google maps, afim de observar e tentar entender a variação das avaliações realizadas pelos usuários da plataforma durante o tempo