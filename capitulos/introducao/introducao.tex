% ----------------------------------------------------------
% Introdução 
% Capítulo sem numeração, mas presente no Sumário
% ----------------------------------------------------------

\chapter[Introdução]{Introdução}
\label{cap:intro}
% \addcontentsline{toc}{chapter}{Introdução}
% pesquisa https://agenciabrasil.ebc.com.br/geral/noticia/2020-05/brasil-tem-134-milhoes-de-usuarios-de-internet-aponta-pesquisa#:~:text=Os%20recursos%20mais%20utilizados%20s%C3%A3o,eletr%C3%B4nico%20(39%25)%20e%20participa%C3%A7%C3%A3o
% https://cetic.br/pt/tics/domicilios/2021/individuos/C6/
% https://books.google.com.br/books?hl=pt-BR&lr=&id=cHYqBF3G3lkC&oi=fnd&pg=PR7&dq=facilidade+no+acesso+a+informacao&ots=g9ggkZ8lS0&sig=-GlYLDNbFKs0q5WYfttQiXkphdI#v=onepage&q=facilidade%20no%20acesso%20a%20informacao&f=false
\begin{comment}
Neste capítulo precisamos:
\begin{itemize}
    \item Introduzir o contexto
    \item Definir o que entendemos como avaliação de hoteis
    \item Apresentar em linhas gerais quais métodos são usados para detectar fake news, qual o estada da arte atual e quais são suas limitações.
    \item Descrever a nossa proposta e objetivos
    \item Descrever a estrutura do relatório
\end{itemize}

\end{comment}

Atualmente o acesso a informação está ficando cada vez mais fácil. Antigamente como fonte de informação recorría-se ao conhecimento de alguém mais experiente ou a conteúdos impressos. Com o avanço da tecnologia estas práticas foram mudando, a prova disso é que nos dias atuais 3 em cada 4 brasileiros acessam a internet. A pesquisa do cetic em 2021 apresenta que 28\% da população entrevistada fez alguma procura por informações referentes a viagens e acomodações, ainda temos um número muito baixo se realizarmos a comparação com procura de informações sobre produtos e serviços em que 57\% dos entrevistados responderam ter realizado alguma consulta.

% https://blog.buscaonibus.com.br/melhores-sites-de-hospedagem-para-sua-viagem/

Sites de viagem como TripAdvisor, Trivago, Booking, AirBnb, são as principais fontes de informações quando temos interresse em realizar uma viagem para um local ainda desconhecido, nesses sites buscamos informações de pessoas que já frequentaram esses locais, essas informações são preenchidas pelos usuários da plataforma e cada plataforma tem o seu modelo de avaliação, que partem de avaliação objetivas por categorias, onde o usuário é questionado por área de interesse de modo a atribuir um valor número como nota para aquela experiência que ele passou, ou mesmo um bloco de texto disponivel para que o usuário possa escrever o que bem quiser sobre a sua estadia no local. Essas plataformas costumam ter alem das avaliações dos usuários, informações que a propria plataforma preenche que customam ser informações de preço por diárias, distâncias de atrações.

Essas informações costumam direcionar o internauta e o auxiliam em sua tomada de decisão, porém qual a validade dessas informações? como essas avaliações variam durante o tempo? esse hotel está tendo avaliações melhores agora do que no mesmo periodo no ano passado? esse local esta sendo avaliado de forma negativa por conta de algum evento atipico? mesmo registrando uma avaliação ruim, o usuário que a submeteu possui outra avaliação registrada no passado? 

Existem diversas questões que as plataformas hoje não expõe de maneira clara o que dificulta a tomada de decisão do internalta, que tem como objetivo escolher um local para usufruir de uma estadia agradável durante o seu período de viagem, ninguém gostaria de reservar um quarto em um hotel e de repente ser surpreendido e acabar descobrindo que o conteúdo das avaliações estavam completamente equivocados, e que na verdade a plataforma estava ocultando as avaliações mais recentes que tinham tendência mais negativa por conta de sua popularidade na plataforma.

\begin{comment}
O presente relatório está estruturado da seguinte forma: o capítulo~\ref{cap:justificativa} apresenta..., o capítulo~\ref{cap:fund_teorica}... O capítulo~\ref{cap:metodologia} ..., o capítulo~\ref{cap:resultados} .... O capítulo~\ref{cap:conclusao} 

Demonstração de citação: o software de análise foi desenvolvido na linguagem Python~\cite{van1995python}, usando as bibliotecas Pandas~\cite{mckinney2010data} e Scikit-learn~\cite{scikit-learn}.
\end{comment}

\section{Objetivo}

O trabalho tem como objetivo realizar a análise de avaliações de hotéis disponibilizadas publicamente em plataformas online, afim de elaborar uma ferramenta que facilite o entendimento dessas avaliações para os internaltas que estiverem procurando informações de hospedagem, explicitando como as avaliações do recinto variam conforme o tempo.

\begin{comment}
    como o sentimento dos usuários que avaliaram o estabelecimento variou durante o tempo, se o recinto está recebendo avaliações com sentimentos mais positivos ou se a tendência é de que as avaliações continuem com sentimentos cada vez mais negativos, e identificiar possíveis mudanças de comportamento, que para esses cenários podem ser justificados por mudanças de equipe, mudanças de políticas internas da empresa ou por uma simples manutenção ou evolução das instalações
    , afim de observar e entender as variações de sentimentos das avaliações realizadas pelos usuários da plataforma distribuidos durante o tempo.
\end{comment}

\section{Justificativa}

Estamos vivendo em um mundo com muitas informações disponiveis e de fácil acesso, porém esta cada vez mais difícil filtrar e escolher as informações que vão fornecer mais insumo para o 